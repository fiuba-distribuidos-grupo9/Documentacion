\documentclass{estilo}
\usepackage[spanish]{babel}
\usepackage{graphicx}
\graphicspath{{img/}}
\usepackage{float}
\usepackage{placeins}
\usepackage{amsmath}        % para los vectores columnas
\usepackage{amsfonts}       % para las negrita de pizarra
\usepackage{amssymb}        % para simbolos matematicos
\usepackage{hyperref}       % para utilizar referencias
\usepackage{multirow}       % para las tablas
\usepackage{dsfont}
\usepackage{listings}
\usepackage{xcolor}
\definecolor{codegreen}{rgb}{0,0.6,0}
\definecolor{codegray}{rgb}{0.5,0.5,0.5}
\definecolor{codepurple}{rgb}{0.58,0,0.82}
\definecolor{backcolour}{rgb}{0.95,0.95,0.92}
\lstdefinestyle{mystyle}{
    backgroundcolor=\color{backcolour},   
    commentstyle=\color{codegreen},
    keywordstyle=\color{magenta},
    numberstyle=\tiny\color{codegray},
    stringstyle=\color{codepurple},
    basicstyle=\ttfamily\footnotesize,
    breakatwhitespace=false,         
    breaklines=true,                 
    captionpos=b,                    
    keepspaces=true,                 
    numbers=left,                    
    numbersep=5pt,                  
    showspaces=false,                
    showstringspaces=false,
    showtabs=false,                  
    tabsize=2
}
\lstset{style=mystyle}

\usepackage{enumitem,multicol,setspace}
\newcounter{subenum}[enumi] % para las multicolumnas
\renewcommand{\thesubenum}{\arabic{subenum}}
\usepackage[nomessages]{fp}
\FPeval\thecolwidth{round(1/4:4)}% Specify number of columns -> column width
\newcommand{\newitem}[1]{%
  \refstepcounter{subenum}%
  \parbox{\dimexpr\thecolwidth\linewidth-.5\columnsep}{%
    \makebox[\labelwidth][r]{(\thesubenum)\hspace*{\labelsep}}%
    #1}\hfill%
}

\usepackage{scalerel,stackengine} % para el sombrero
\stackMath
\newcommand\rhat[1]{%
\savestack{\tmpbox}{\stretchto{%
  \scaleto{%
    \scalerel*[\widthof{\ensuremath{#1}}]{\kern-.6pt\bigwedge\kern-.6pt}%
    {\rule[-\textheight/2]{1ex}{\textheight}}%WIDTH-LIMITED BIG WEDGE
  }{\textheight}% 
}{0.5ex}}%
\stackon[1pt]{#1}{\tmpbox}%
}
\parskip 1ex

\usepackage{mathtools}      % floor y ceil
\DeclarePairedDelimiter\ceil{\lceil}{\rceil}
\DeclarePairedDelimiter\floor{\lfloor}{\rfloor} 

\usepackage[style=authoryear-comp]{biblatex}

\begin{document}
\maketitle

\newpage
\justifying{
\input{tex/tabla de contenidos}
\newpage
\section{Introducción}

El presente documento describe el diseño de un sistema distribuido para el análisis de datos de una cadena de cafeterías en Malasia. 
El objetivo principal es procesar grandes volúmenes de información transaccional, de clientes, de sucursales y de productos, 
para obtener métricas clave que apoyen la toma de decisiones de negocio.

De acuerdo con los requisitos existentes, el sistema debe permitir responder las siguientes consultas:

\begin{enumerate}
    \item Listado de transacciones (ID y monto) realizadas entre los años 2024 y 2025, en el rango horario de 06:00 a 23:00, con un monto total mayor o igual a 75.
    \item Identificación de los productos más vendidos (nombre y cantidad) y los que más ganancias han generado (nombre y monto) para cada mes de 2024 y 2025.
    \item Cálculo del \textit{Total Payment Value} (TPV) por cada semestre en 2024 y 2025, discriminado por sucursal, considerando sólo transacciones entre 06:00 y 23:00.
    \item Obtención de la fecha de cumpleaños de los tres clientes con mayor cantidad de compras en 2024 y 2025, para cada sucursal.
\end{enumerate}

Además de los requerimientos funcionales, el sistema deberá cumplir con los siguientes requisitos no funcionales:

\begin{itemize}
    \item Optimización para entornos multicomputadoras, asegurando la escalabilidad ante el crecimiento de datos.
    \item Inclusión de un \textit{middleware} que abstraiga la comunicación entre nodos mediante grupos.
    \item Ejecución única del procesamiento con capacidad de \textit{graceful quit} frente a señales de terminación (SIGTERM).
\end{itemize}

El diseño se realizará bajo un enfoque de arquitectura distribuida, 
buscando flexibilidad, robustez y capacidad de escalar en entornos reales de procesamiento de datos.

\newpage
\section{Definición Técnica del Sistema Distribuido}

El sistema distribuido propuesto permite al usuario interactuar mediante una consola, 
desde la cual los usuarios envían las instrucciones y datasets necesarios para el procesamiento, y a cambio reciben como salida de la misma el resultado de las consultas realizadas. 
La infraestructura se implementará sobre contenedores Docker, los cuales emulan 
múltiples nodos computacionales. Cada nodo se especializa en la ejecución de una operación 
determinada, tales como filtrado, mapeo, reducción, ordenamiento o combinación de datos.  

El sistema recibe como entrada un conjunto de archivos con la información a procesar:

\begin{itemize}
    \item \texttt{menu\_items.csv}
    \item \texttt{payment\_methods.csv}
    \item \texttt{stores.csv}
    \item Múltiples archivos \texttt{transaction\_items.csv}
    \item Múltiples archivos \texttt{transactions.csv}
    \item Múltiples archivos \texttt{users.csv}
    \item \texttt{vouchers.csv}
\end{itemize}

Estos archivos son distribuidos entre los distintos nodos según lo que requiera cada operación. 
La comunicación entre el nodo coordinador y los nodos de procesamiento se gestiona mediante 
un \textit{middleware}, el cual se describe en detalle en una sección posterior.

\subsection{Operaciones Distribuidas}

Las operaciones fundamentales soportadas por el sistema son las siguientes:

\begin{enumerate}
    \item \textbf{Filter}: Permite seleccionar subconjuntos de datos en función de condiciones 
    específicas. Una misma instancia puede aplicar diferentes filtros según el criterio 
    definido en la consulta (por ejemplo, filtrar por fechas, montos o rangos horarios).
    
    \item \textbf{Map}: Transforma los datos de entrada generando nuevas columnas o 
    reformateando la información existente. Ejemplo: Agregar un campo con el mes y año a partir 
    de una fecha, o calcular un contador auxiliar.
    
    \item \textbf{Reduce}: Agrupa datos por una clave determinada y aplica una función de 
    agregación (como sumas, conteos o acumulaciones). Existen múltiples variantes, dependiendo 
    de qué métrica se busque consolidar.
    
    \item \textbf{Join}: Combina registros de diferentes datasets en función de un campo común, 
    permitiendo enriquecer la información (ejemplo: Unir transacciones con usuarios para obtener 
    la fecha de nacimiento del cliente).
    
    \item \textbf{SortBy}: Ordena los registros de acuerdo con uno o más criterios, ya sea de 
    manera ascendente o descendente. Es fundamental para identificar los elementos más 
    significativos en cada consulta (por ejemplo, productos más vendidos).
\end{enumerate}

Cada nodo de operación está diseñado para ser genérico y flexible: Puede ejecutar cualquier variante de las funciones mencionadas según el rol específico del nodo, siendo el sistema coordinador quien le envía los datos y la 
configuración necesaria para que realice la tarea solicitada.
\newpage
\section{Implementación del Usuario}

La interacción de los usuarios con el sistema distribuido se realizará mediante un esquema 
secuensial. Para cada usuario que se conecte, se respetera una secuencia específica de ejecución.

\begin{itemize}
    \item \textbf{Etapa de envío de información:} En esta etapa el usuario envía toda la información en base a la que el
    sistema debe operar. Esto incluye todos los datasets a procesar para resolver las 4 consultas.
    
    \item \textbf{Etapa de espera:} El cliente queda a la espera de que el sistema distribuido procese la información y genere las respuestas
    a las consultas.

    \item \textbf{Etapa de recepción de resultados:} Una vez que el sistema ha procesado la información, el usuario recibe las respuestas a las consultas
    solicitadas.
\end{itemize}

De esta manera se logra una interacción segura y escalable.

La arquitectura está diseñada para soportar múltiples usuarios en simultáneo, escalando el número de conexiones establecidas proporcionalmente a la cantidad de clientes activos.

\section{Mecanismo de Optimización del Uso de Recursos}

Cada computadora que participa como nodo de procesamiento en el sistema distribuido 
estará dedicada a ejecutar una operación determinada (por ejemplo, filtrado o reducción). 

Si bien en una primera versión se teorizó la posibilidad de generar varios procesos por 'CPU' para buscar algún tipo de optimización del rendimiento de cada computadora, luego de la primera entrega con el corrector se definió que no resultaría necesario para la implementación de la resolución.

Por lo tanto, se descarta esta mejora para esta versión del sistema, pero se deja como recomendación por si se realiza una actualización a futuro en búsqueda de optimizar tiempos de procesamiento de consultas.

De ser así, se debe recordar que al implementar la paralelización, se debe hacer mediante 
\textbf{multiprocesos} y no mediante \textbf{multithreading}. 
La justificación de esta elección radica en que Python presenta limitaciones para 
tareas de cómputo intensivo debido al \textit{Global Interpreter Lock} (GIL).

La librería de \texttt{multiprocessing} permite crear procesos independientes que aprovechan 
mejor las capacidades de CPUs con múltiples núcleos.
\section{Uso de Archivos Intermedios}

En el diseño del sistema se contempla la generación de archivos intermedios durante 
el procesamiento de las consultas. Esta decisión se fundamenta en un aspectos clave:

\textbf{Respaldo de información:} Los archivos intermedios actúan como puntos de control que facilitan futuras implementaciones de tolerancia a fallas. 
En caso de interrupciones, será posible retomar el procesamiento desde un estado parcial previamente almacenado.

El uso de archivos intermedios, si bien introduce un costo adicional de I/O, 
aumenta la robustez del sistema y habilita mejoras posteriores en términos de 
recuperación y confiabilidad.

Para esta entrega del sistema todavía no están implementados de manera funcional, pero si se deja esta consideración para luego documentar su funcionamiento cuando corresponda.
}

\newpage
\justifying{
\section{Descripción de las Consultas}

El sistema debe ser capaz de responder a las siguientes consultas, de acuerdo con la lógica 
planteada en el enunciado. A continuación, se detalla el paso a paso de cada una de ellas, 
indicando las tablas necesarias, las columnas relevantes y la secuencia de operaciones 
distribuidas.

\subsection{Consulta 1: Transacciones filtradas}
\textbf{Objetivo:} Obtener las transacciones (ID y monto) realizadas durante 2024 y 2025, 
entre las 06:00 y las 23:00 horas, con un monto total mayor o igual a 75.

\begin{itemize}
    \item \textbf{Tablas requeridas:} \texttt{transactions}.
    \item \textbf{Columnas utilizadas:} \texttt{transaction\_id}, \texttt{created\_at}, \texttt{final\_amount}.
    \item \textbf{Pasos:}
    \begin{enumerate}
        \item Filtrar transacciones entre 2024 y 2025, en el rango horario indicado.
        \item Filtrar transacciones con monto mayor o igual a 75.
        \item Generar el reporte y devolverlo al usuario.
    \end{enumerate}
\end{itemize}

\subsection{Consulta 2: Productos más vendidos y más rentables}
\textbf{Objetivo:} Identificar, para cada mes de 2024 y 2025, los productos más vendidos (nombre y cantidad) 
y los productos que más ganancias generaron (nombre y monto).

\begin{itemize}
    \item \textbf{Tablas requeridas:} \texttt{transaction\_items}, \texttt{menu\_items}.
    \item \textbf{Columnas utilizadas:} 
    \begin{itemize}
        \item De \texttt{transaction\_items}: \texttt{item\_id}, \texttt{created\_at}, \texttt{quantity}, \texttt{subtotal}.
        \item De \texttt{menu\_items}: \texttt{item\_id}, \texttt{item\_name}.
    \end{itemize}
    \item \textbf{Pasos:}
    \begin{enumerate}
        \item Filtrar los \texttt{transaction\_items} de 2024 y 2025.
        \item Generar una nueva columna \texttt{year\_month} a partir de la fecha.
        \item Para productos más vendidos:
        \begin{enumerate}
            \item Calcular contador de cantidad (\texttt{item\_counter}).
            \item Reducir por clave \texttt{year\_month, item\_id} sumando las cantidades.
            \item Ordenar por cantidad descendente.
        \end{enumerate}
        \item Para productos más rentables:
        \begin{enumerate}
            \item Calcular monto acumulado (\texttt{total\_amount\_counter}).
            \item Reducir por clave \texttt{year\_month, item\_id} sumando subtotales.
            \item Ordenar por monto descendente.
        \end{enumerate}
        \item Unir con \texttt{menu\_items} para obtener nombres de los productos.
        \item Generar el reporte y devolverlo al usuario.
    \end{enumerate}
\end{itemize}

\subsection{Consulta 3: TPV por semestre y sucursal}
\textbf{Objetivo:} Calcular el \textit{Total Payment Value} (TPV) por cada semestre de 2024 y 2025, 
para cada sucursal, considerando únicamente transacciones realizadas entre 06:00 y 23:00.

\begin{itemize}
    \item \textbf{Tablas requeridas:} \texttt{transactions} y \texttt{stores}.
    \item \textbf{Columnas utilizadas:} \texttt{store\_id}, \texttt{created\_at}, \texttt{final\_amount}.
    \item \textbf{Pasos:}
    \begin{enumerate}
        \item Filtrar transacciones entre 2024 y 2025, dentro del rango horario.
        \item Generar nueva columna \texttt{year\_semester} a partir de la fecha.
        \item Reducir por clave \texttt{year\_semester, store\_id} sumando los montos finales.
        \item Unir con \texttt{stores} para obtener nombres de las sucursales.
        \item Generar el reporte y devolverlo al usuario.
    \end{enumerate}
\end{itemize}

\subsection{Consulta 4: Clientes más frecuentes y cumpleaños}
\textbf{Objetivo:} Obtener la fecha de cumpleaños de los tres clientes con mayor número de compras 
durante 2024 y 2025, discriminado por sucursal.

\begin{itemize}
    \item \textbf{Tablas requeridas:} \texttt{transactions}, \texttt{users} y \texttt{stores}.
    \item \textbf{Columnas utilizadas:} 
    \begin{itemize}
        \item De \texttt{transactions}: \texttt{user\_id}, \texttt{store\_id}, \texttt{created\_at}.
        \item De \texttt{users}: \texttt{user\_id}, \texttt{birthdate}.
    \end{itemize}
    \item \textbf{Pasos:}
    \begin{enumerate}
        \item Filtrar transacciones de 2024 y 2025.
        \item Generar clave combinada \texttt{store\_user} con \texttt{store\_id} y \texttt{user\_id}.
        \item Calcular contador de compras (\texttt{buys\_counter}).
        \item Reducir por clave \texttt{store\_user} sumando los contadores.
        \item Ordenar en forma descendente por número de compras.
        \item Seleccionar los tres usuarios con más compras por sucursal.
        \item Unir con \texttt{users} para obtener las fechas de nacimiento.
        \item Unir con \texttt{stores} para obtener nombres de las sucursales.
        \item Generar el reporte y devolverlo al usuario.
    \end{enumerate}
\end{itemize}

}

\newpage
\justifying{
\section{Vista de Escenarios}

\subsection{Casos de uso}

Para modelar la interacción principal entre los actores y el sistema, se presenta un diagrama de casos de uso. Este diagrama identifica al actor principal/client, denominado "Cafetería", y las cuatro funcionalidades clave que el sistema debe proveer, correspondiendo directamente a las consultas de negocio definidas en los requerimientos.

\FloatBarrier
\begin{figure}[H]
  \centering
  \includegraphics[width=\linewidth, height=.82\textheight, keepaspectratio]{casosdeuso.png}
  \caption{Casos de Uso}
  \label{fig:casos_de_uso}
\end{figure}

\newpage
\section{Vista de Procesos}

% ================== DIAGRAMAS DE ACTIVIDADES ================== 

\subsection{Diagramas de Actividades}

Los diagramas de actividad describen el flujo de trabajo lógico para cada una de las consultas funcionales. A continuación, se presenta un diagrama para cada consulta, detallando la secuencia de acciones y las decisiones desde el inicio de la solicitud hasta la presentación de los resultados finales.

\subsubsection{Diagrama de Actividades de la primera consulta}

\FloatBarrier
\begin{figure}[H]
  \centering
  \includegraphics[width=\linewidth, height=.60\textheight, keepaspectratio]{actividad1.png}
  \caption{Actividad de la primera consulta}
  \label{fig:actividad1}
\end{figure}

\subsubsection{Diagrama de Actividades de la segunda consulta}

\FloatBarrier
\begin{figure}[H]
  \centering
  \includegraphics[width=\linewidth, height=.75\textheight, keepaspectratio]{actividad2.png}
  \caption{Actividad de la segunda consulta}
  \label{fig:actividad2}
\end{figure}

\subsubsection{Diagrama de Actividades de la tercera consulta}

\FloatBarrier
\begin{figure}[H]
  \centering
  \includegraphics[width=\linewidth, height=.75\textheight, keepaspectratio]{actividad3.png}
  \caption{Actividad de la tercera consulta}
  \label{fig:actividad3}
\end{figure}

\subsubsection{Diagrama de Actividades de la cuarta consulta}

\FloatBarrier
\begin{figure}[H]
  \centering
  \includegraphics[width=\linewidth, height=.75\textheight, keepaspectratio]{actividad4.png}
  \caption{Actividad de la cuarta consulta}
  \label{fig:actividad4}
\end{figure}


% ================== DIAGRAMAS DE SECUENCIA ==================

\newpage

\subsection{Diagramas de Secuencia}

Para ilustrar la interacción temporal y el intercambio de mensajes entre los distintos componentes del sistema, se utilizan los diagramas de secuencia. Cada diagrama modela la ejecución de una consulta, mostrando cómo el Client Process envía lotes de datos al Server, el cual coordina las operaciones distribuidas entre los nodos especializados como Filter, GroupBy y ReduceBy hasta obtener y devolver el reporte final.

\subsubsection{Diagrama de Secuencia de la primera consulta}

\vspace{1cm}

\FloatBarrier
\begin{figure}[H]
  \centering
  \includegraphics[width=\linewidth, height=.82\textheight, keepaspectratio]{img/Vista de Procesos - Diagrama Secuencia Query 1.drawio.png}
  \caption{Secuencia de la primera consulta}
  \label{fig:secuencia1}
\end{figure}

\subsubsection{Diagrama de Secuencia de la segunda consulta}

\vspace{1cm}

\FloatBarrier
\begin{figure}[H]
  \centering
  \includegraphics[width=\linewidth, height=.35\textheight, keepaspectratio]{img/Vista de Procesos - Diagrama Secuencia Query 2.drawio.png}
  \caption{Secuencia de la segunda consulta}
  \label{fig:secuencia2}
\end{figure}

\subsubsection{Diagrama de Secuencia de la tercera consulta}

\vspace{1cm}

\FloatBarrier
\begin{figure}[H]
  \centering
  \includegraphics[width=\linewidth, height=.35\textheight, keepaspectratio]{img/Vista de Procesos - Diagrama Secuencia Query 3.drawio.png}
  \caption{Secuencia de la tercera consulta}
  \label{fig:secuencia3}
\end{figure}

\subsubsection{Diagrama de Secuencia de la cuarta consulta}

\vspace{1cm}

\FloatBarrier
\begin{figure}[H]
  \centering
  \includegraphics[width=\linewidth, height=.82\textheight, keepaspectratio]{img/Vista de Procesos - Diagrama Secuencia Query 4.drawio.png}
  \caption{Secuencia de la cuarta consulta}
  \label{fig:secuencia4}
\end{figure}
\newpage
\section{Vista Física}

\subsection{Diagrama de Robustez}

El diagrama de robustez forma parte de la vista lógica y sirve como puente entre los casos de uso y el diseño detallado. Este diagrama permite identificar y organizar los principales elementos del sistema, diferenciando entre actores externos, límites de la aplicación y las entidades o controladores que gestionan la lógica de negocio. 

\FloatBarrier
\begin{figure}[H]
  \centering
  \rotatebox{270}{%
    \includegraphics[
      width=.75\textheight,
      keepaspectratio
    ]{img/robustez.png}%
  }
  \caption{Diagrama de Robustez - Completo}
  \label{fig:robustez}
\end{figure}

\newpage

\subsection{Diagrama de Robustez - Primera Consulta}

\FloatBarrier
\begin{figure}[H]
  \centering
  \rotatebox{270}{%
    \includegraphics[
      width=.85\textheight,
      keepaspectratio
    ]{img/robustez1.png}%
  }
  \caption{Diagrama de Robustez - Primera Consulta}
  \label{fig:robustez}
\end{figure}

\newpage

\subsection{Diagrama de Robustez - Segunda Consulta}

\FloatBarrier
\begin{figure}[H]
  \centering
  \rotatebox{270}{%
    \includegraphics[
      width=.85\textheight,
      keepaspectratio
    ]{img/robustez2.png}%
  }
  \caption{Diagrama de Robustez - Segunda Consulta}
  \label{fig:robustez}
\end{figure}

\newpage

\subsection{Diagrama de Robustez - Tercera Consulta}

\FloatBarrier
\begin{figure}[H]
  \centering
  \rotatebox{270}{%
    \includegraphics[
      width=.85\textheight,
      keepaspectratio
    ]{img/robustez3.png}%
  }
  \caption{Diagrama de Robustez - Tercera Consulta}
  \label{fig:robustez}
\end{figure}

\newpage

\subsection{Diagrama de Robustez - Cuarta Consulta}

\FloatBarrier
\begin{figure}[H]
  \centering
  \rotatebox{270}{%
    \includegraphics[
      width=.85\textheight,
      keepaspectratio
    ]{img/robustez4.png}%
  }
  \caption{Diagrama de Robustez - Cuarta Consulta}
  \label{fig:robustez}
\end{figure}

\newpage

\subsection{Escalabilidad de las Operaciones}

Dentro de las operaciones del sistema distribuido, existen tres que admiten escalabilidad 
en la cantidad de nodos de procesamiento:

\begin{itemize}
    \item \textbf{Cleaner}
    \item \textbf{Filter}
    \item \textbf{Join}
\end{itemize}

Como decisión de diseño, se definió que en los casos donde el sistema cuente con una 
disposición de $N$ nodos de alguno de los tres tipos mencionados, comunicándose con 
$M$ nodos de la misma categoría, cada uno de los $M$ nodos receptores tendrá su propia cola.  
Dichas colas recibirán los mensajes enviados por los $N$ nodos anteriores.  
Cuando un nodo de la segunda capa reciba en su cola un mensaje indicador de finalización, 
podrá tener la certeza de que no recibirá más información desde esos nodos, 
y podrá trasladar dicho aviso hacia etapas posteriores en caso de ser necesario.  

Para garantizar este comportamiento, al iniciar la ejecución del sistema todos los nodos 
conocerán la topología de sus vecinos, lo que permitirá coordinar de forma correcta 
la comunicación y el flujo de datos durante el procesamiento.

A continuación, se presenta un esquema ilustrativo de la topología entre $N$ nodos 
\textit{Filter} y $M$ nodos \textit{Join}, junto con sus respectivas $M$ colas:

\FloatBarrier
\begin{figure}[H]
  \centering
  \rotatebox{0}{%
    \includegraphics[
      width=.60\textheight,
      keepaspectratio
    ]{img/nodos_escalabilidad.png}%
  }
  \caption{Esquema representativo de la comunicación de 'N' nodos con 'M' colas de otros 'M' nodos}
  \label{fig:robustez}
\end{figure}

\bigskip

Asimismo, se adoptó la decisión de diseño de no escalar con múltiples computadoras los 
\textit{Join} que se realizan con las tablas de \textit{Menú} y de \textit{Stores}, 
ya que ambas son lo suficientemente pequeñas como para cargarse íntegramente en memoria.  
De esta forma, se mantiene la tabla más pequeña residente en memoria y se la utiliza 
para realizar \textit{joins} eficientes a medida que llegan las transacciones, 
maximizando así la performance.

Tampoco se escalaran con múltiples computadoras las operaciones 'Map', 'Reduce', 'Sort' y 'Output Builder'. Ya que son 'Stateful' y su correcto funcionamiento e implementación se ve muy beneficiado de ejecutarse en una sola instancia de computación, sin perder un rendimiento perceptible.

En contraste, para el caso de los \textit{Join} con la tabla de \textit{Usuarios} 
sí se habilita la escalabilidad a múltiples nodos, dado que el volumen de datos 
puede ser significativamente mayor.  
En este escenario, el listado de usuarios se dividirá entre los $M$ nodos mediante 
una función de \textit{hash}.  
Los nodos que depositen la información de transacciones en las colas aplicarán 
la misma función de \textit{hash} sobre los identificadores de usuario, determinando 
de manera directa a qué cola (y por ende a qué nodo) deben dirigir cada registro 
para asegurar que el \textit{join} se realice correctamente.

\subsection{Diagrama de Despliegue}

Dentro de la vista física realizamos el diagrama de despliegue, el cual ilustra la topología del sistema distribuido en términos de sus nodos computacionales. El diagrama muestra los distintos grupos de nodos especializados (Data Cleaners, Filters, Maps, Joins, etc.) y cómo se interconectan a través de un componente central: el MOM Broker. Esta arquitectura facilita la comunicación asincrónica y el desacoplamiento entre los procesos.

\vspace{1cm}

\FloatBarrier
\begin{figure}[H]
  \centering
  \includegraphics[width=\linewidth, height=.82\textheight, keepaspectratio]{img/Vista Física - Diagrama de Despliegue.drawio.png}
  \caption{Diagrama de Despliegue}
  \label{fig:diagrama_de_despliegue}
\end{figure}


\newpage
\section{Vista de Desarrollo}

\subsection{Diagrama de Paquetes}

La vista de desarrollo organiza el sistema en un conjunto de paquetes lógicos para promover una alta cohesión y un bajo acoplamiento. El diagrama muestra los componentes clave: el paquete shared contiene la lógica de comunicación común, como el MQConnectionHandler; el paquete controllers agrupa las distintas operaciones distribuidas (filtros, mapeos, etc.); y los paquetes client y server definen los puntos de entrada de la aplicación.

\vspace{1cm}

\FloatBarrier
\begin{figure}[H]
  \centering
  \includegraphics[width=\linewidth, height=.82\textheight, keepaspectratio]{img/Vista de Desarrollo - Diagrama de Paquetes.drawio.png}
  \caption{Vista de Desarrollo - Diagrama de Paquetes}
  \label{fig:diagrama_de_paquetes}
\end{figure}

\newpage
\section{Vista Lógica}

\subsection{Diagrama de Clases}

La vista lógica detalla la estructura estática del sistema a través de diagramas de clases. El diseño se basa en la herencia y la abstracción para maximizar la reutilización de código. Como se observa en los siguientes diagramas, se define una clase base abstracta Controller de la cual heredan las distintas operaciones especializadas (Filter, Count, Sum, etc.), permitiendo que el sistema maneje diferentes tipos de tareas de manera uniforme y extensible.

\subsubsection{Diagrama de Clases - DataCleaner}

\FloatBarrier
\begin{figure}[H]
  \centering
  \includegraphics[width=\linewidth, height=.82\textheight, keepaspectratio]{img/Vista Lógica - Diagrama de Clases-DataCleaner.drawio.png}
  \caption{Clase DataCleaner}
  \label{fig:data_cleaner_class}
\end{figure}

\subsubsection{Diagrama de Clases - Filter}

\FloatBarrier
\begin{figure}[H]
  \centering
  \includegraphics[width=\linewidth, height=.82\textheight, keepaspectratio]{img/Vista Lógica - Diagrama de Clases-Filter.drawio.png}
  \caption{Clase Filter}
  \label{fig:filter_class}
\end{figure}

\subsubsection{Diagrama de Clases - Count y Sum}

\FloatBarrier
\begin{figure}[H]
  \centering
  \includegraphics[width=\linewidth, height=.82\textheight, keepaspectratio]{img/Vista Lógica - Diagrama de Clases-Count & Sum.drawio.png}
  \caption{Clases Count y Sum}
  \label{fig:count_and_sum_classes}
\end{figure}

\subsubsection{Diagrama de Clases - SortDesc}

\FloatBarrier
\begin{figure}[H]
  \centering
  \includegraphics[width=\linewidth, height=.82\textheight, keepaspectratio]{img/Vista Lógica - Diagrama de Clases-SortDesc.drawio.png}
  \caption{Clase SortDesc}
  \label{fig:sort_desc_class}
\end{figure}

\subsubsection{Diagrama de Clases - Join}

\FloatBarrier
\begin{figure}[H]
  \centering
  \includegraphics[width=\linewidth, height=.82\textheight, keepaspectratio]{img/Vista Lógica - Diagrama de Clases-Join.drawio.png}
  \caption{Clase Join}
  \label{fig:join_class}
\end{figure}

\subsubsection{Diagrama de Clases - OutputBuilder}

\FloatBarrier
\begin{figure}[H]
  \centering
  \includegraphics[width=\linewidth, height=.82\textheight, keepaspectratio]{img/Vista Lógica - Diagrama de Clases-OutputBuilder.drawio.png}
  \caption{Clase OutputBuilder}
  \label{fig:output_builder_class}
\end{figure}
}

\newpage
\justifying{
\section{Diagrama de Grafo Acíclico Dirigido (DAG) - Completo}

El flujo de procesamiento completo para resolver las consultas se modela como un Grafo Acíclico Dirigido (DAG). Este diagrama visualiza las dependencias entre las distintas operaciones distribuidas. Cada nodo en el grafo representa una tarea (ej. Clean, Filter by Year, Count Items), y las aristas indican el flujo de datos desde una etapa hacia la siguiente, mostrando cómo se combinan las distintas fuentes de datos para producir los resultados finales (Q1, Q2, Q3, Q4).

\subsection{Diagrama de Grafo Acíclico Dirigido (DAG) - Completo}

\FloatBarrier
\begin{figure}[H]
  \centering
  \includegraphics[width=\linewidth, height=.82\textheight, keepaspectratio]{img/DAG.drawio.png}
  \caption{Diagrama de Grafo Acíclico Dirigido (DAG) - Completo}
  \label{fig:dag}
\end{figure}

\newpage

\subsection{Diagrama de Grafo Acíclico Dirigido (DAG) - Primera consulta}

\FloatBarrier
\begin{figure}[H]
  \centering
  \includegraphics[width=\linewidth, height=.82\textheight, keepaspectratio]{img/DAG1.png}
  \caption{Diagrama de Grafo Acíclico Dirigido (DAG) - Primera consulta}
  \label{fig:dag}
\end{figure}

\newpage

\subsection{Diagrama de Grafo Acíclico Dirigido (DAG) - Segunda consulta}

\FloatBarrier
\begin{figure}[H]
  \centering
  \includegraphics[width=\linewidth, height=.82\textheight, keepaspectratio]{img/DAG2.png}
  \caption{Diagrama de Grafo Acíclico Dirigido (DAG) - Segunda consulta}
  \label{fig:dag}
\end{figure}

\newpage

\subsection{Diagrama de Grafo Acíclico Dirigido (DAG) - Tercera consulta}

\FloatBarrier
\begin{figure}[H]
  \centering
  \includegraphics[width=\linewidth, height=.82\textheight, keepaspectratio]{img/DAG3.png}
  \caption{Diagrama de Grafo Acíclico Dirigido (DAG) - Tercera consulta}
  \label{fig:dag}
\end{figure}

\newpage

\subsection{Diagrama de Grafo Acíclico Dirigido (DAG) - Cuarta consulta}

\FloatBarrier
\begin{figure}[H]
  \centering
  \includegraphics[width=\linewidth, height=.82\textheight, keepaspectratio]{img/DAG4.png}
  \caption{Diagrama de Grafo Acíclico Dirigido (DAG) - Cuarta consulta}
  \label{fig:dag}
\end{figure}
}

\newpage
\justifying{
\section{Middleware}

El middleware interno constituye el pilar fundamental de la comunicación en el sistema \textit{Coffee Shop Analysis}, siendo el responsable de orquestar la interacción entre el servidor central y los múltiples nodos de procesamiento. Su arquitectura se basa en un modelo de Middleware Orientado a Mensajes (MOM), implementado mediante la herramienta \textbf{RabbitMQ}, que actúa como bróker central y mediador confiable entre los distintos procesos distribuidos, tal como se visualiza en el Diagrama de Despliegue.  

La adopción de un middleware de este tipo permite abstraer completamente la complejidad de la comunicación en red, ocultando al programador los detalles de bajo nivel relacionados con sockets, serialización de datos o control de concurrencia. En lugar de conexiones rígidas punto a punto, los componentes se comunican a través de colas y \emph{exchanges} definidos estratégicamente para cada escenario. Este enfoque incrementa notablemente la flexibilidad del sistema, ya que la incorporación de un nuevo nodo de procesamiento, o la eliminación de uno existente, no requiere modificar la lógica de negocio del resto de los participantes.  

Otro aspecto central es la asincronía. Gracias al middleware, los nodos no dependen de la disponibilidad inmediata de sus contrapartes, sino que pueden depositar mensajes en una cola para que el receptor los procese cuando corresponda. Esto aporta robustez frente a picos de carga y tolerancia a fallos parciales, ya que los datos no se pierden aunque un nodo se encuentre momentáneamente inactivo. Asimismo, las colas actúan como un mecanismo de \textit{buffering}, desacoplando el ritmo de producción y consumo de datos entre los distintos nodos.

La lógica de conexión se implementa en el componente \texttt{MQConnectionHandler}, ubicado en el paquete \texttt{shared}, que centraliza la configuración de colas, \emph{exchanges} y canales de comunicación. De este modo, se asegura un manejo uniforme y estandarizado de las operaciones de envío y recepción de mensajes. Sobre esta capa de mensajería se diseñó además un protocolo de aplicación específico que define la estructura de los mensajes para el envío de datasets, la transmisión de resultados y el manejo de errores, lo que permite garantizar la correcta interpretación de la información en todas las etapas del procesamiento.

Finalmente, para interactuar con el middleware se emplearon las interfaces provistas por la cátedra, complementadas con la implementación de pruebas unitarias exhaustivas que validaron su correcto funcionamiento bajo diferentes escenarios de carga. De esta manera, el middleware cumple con el requisito planteado en el enunciado de abstraer la comunicación entre los nodos del sistema distribuido, aportando una base sólida para la escalabilidad, el desacoplamiento y la mantenibilidad del sistema en su conjunto.

\newpage

\subsection*{Esquemas de comunicación}

Durante el desarrollo se identificaron distintos patrones de comunicación, y para cada uno se diseñó una solución particular apoyada en las herramientas de RabbitMQ. A continuación, se describen los principales casos implementados

\subsubsection*{Comunicación mediante colas dedicadas}

En los nodos escalables (Para esta versión todos los nodos de nuestro Sistema Distribuido son escalables), se definió una cola por cada instancia de nodo. Esto asegura la ausencia de \textit{race conditions} y permite direccionar la información de forma controlada, garantizando el funcionamiento correcto y balanceado del sistema.

Para decidir a qué cola enviar cada batch de datos se utilizaron dos estrategias:
\begin{enumerate}
    \item \textbf{Round Robin:} Cada emisor mantiene un contador que se incrementa en cada envío, distribuyendo los lotes de datos de forma equitativa entre las colas disponibles. Una vez alcanzado el último nodo, el contador retorna al primero.  
    Esta técnica fue utilizada en situaciones como:
    \begin{itemize}
        \item Servidor $\rightarrow$ Cleaners.
        \item Cleaners $\rightarrow$ Filters.
        \item Filters $\rightarrow$ Filters.
        \item Filters $\rightarrow$ Output Builder.
        \item Join Users $\rightarrow$ Join Stores.
        \item Filters $\rightarrow$ Map Year Month / Map Year Semester.
    \end{itemize}

    \item \textbf{Hashing por clave (Sharding):} Se empleó en los casos en que los datos debían ser dirigidos a un nodo específico según una clave.
    Por ejemplo, en la comunicación de:

    \begin{itemize}
        \item Cleaners de Usuarios $\rightarrow$ Joins de Usuarios.
    \end{itemize}
    
    La asignación de transacciones a los nodos de Join se resolvió aplicando una función hash sobre el ID de usuario.
    
    Concretamente, se utilizó el resto de la división entera del ID por la cantidad de nodos shardeados, lo que asegura que todas las transacciones de un mismo usuario lleguen al mismo nodo de Join.
\end{enumerate}

\begin{figure}[H]
    \centering
    \includegraphics[width=0.5\linewidth]{img/comunicacion1.png}
    \caption{Esquema de comunicación mediante colas dedicadas.}
\end{figure}

\newpage

\subsubsection*{Comunicación mediante exchanges}

En situaciones donde la misma información debía ser replicada en múltiples colas, se utilizó un \textbf{exchange} de RabbitMQ. Este patrón resultó especialmente útil en la \textbf{Query 2}, donde se debía dividir la información para dos subconsultas distintas: los ítems más vendidos y los que generaron mayor facturación.  

En este caso, el exchange distribuye la información procesada por el nodo \textit{Map Year Month} hacia las colas de:
\begin{itemize}
    \item Count Items.
    \item Sum Items.
\end{itemize}

\begin{figure}[H]
    \centering
    \includegraphics[width=0.4\linewidth]{img/comunicacion2.png}
    \caption{Esquema de comunicación mediante exchanges.}
\end{figure}

\subsection*{Detección y propagación de EOF}

En la nueva versión del sistema, con soporte \textbf{MultiClient}, cada cliente opera dentro de su propio \emph{contexto de sesión}, identificado por un \texttt{session\_id} (UUID).  
Esto implica que la detección y propagación de los mensajes de fin de flujo (\texttt{EOF}) debe realizarse de forma independiente por sesión, garantizando el cierre correcto de cada flujo de datos asociado a un cliente específico.

El mecanismo general conserva la misma lógica que en la versión de un solo cliente, pero se amplía para manejar múltiples sesiones concurrentes:

\begin{itemize}
    \item Cada nodo mantiene un \textbf{registro por sesión} de los emisores de los que espera recibir \texttt{EOF}. Es decir, no sólo sabe qué nodos le preceden, sino también a qué \emph{session\_id} pertenece cada flujo.
    \item Por cada \texttt{session\_id}, el nodo espera recibir la cantidad exacta de mensajes \texttt{EOF} correspondientes a sus emisores aguas arriba.
    \item Cuando un nodo ha recibido todos los \texttt{EOF} esperados para una sesión dada, puede concluir que no habrá más datos de entrada para esa sesión en particular.
    \item En ese momento, el nodo genera y propaga su propio \texttt{EOF} —incluyendo el mismo \texttt{session\_id}— hacia todos los nodos que le suceden, utilizando el mismo esquema de enrutamiento:
    \begin{itemize}
        \item Enviando a una cola directa (1 a 1).
        \item Replicando en un \emph{exchange} compartido.
        \item Distribuyendo por \emph{Round Robin} entre varias colas consumidoras.
    \end{itemize}
    \item Este proceso ocurre de forma independiente para cada sesión activa, permitiendo que distintos clientes finalicen sus flujos en momentos diferentes sin interferir entre sí.
\end{itemize}

De esta manera, el sistema logra una \textbf{propagación de EOF por sesión}, garantizando que cada cliente tenga un cierre completo y ordenado de su pipeline de procesamiento.  
Además, los nodos pueden liberar recursos por sesión una vez propagado el último \texttt{EOF}, mejorando la eficiencia y evitando bloqueos globales entre flujos concurrentes.

\newpage
\section{Mensajes y Protocolos}
\label{sec:mensajes-y-protocolos}

\subsection{Protocolo}
\label{subsec:protocolo}

Para la implementación del sistema distribuido se desarrolló un protocolo de
comunicación específico, adaptado a las necesidades del presente trabajo.

Este protocolo define las reglas de interacción entre el nodo coordinador,
los nodos de procesamiento y los clientes del sistema. La comunicación se basa
en el intercambio de mensajes \emph{auto\-contenidos} con un encabezado de tipo fijo
y un \emph{payload} estructurado. Los mensajes expresan operaciones, datos a
procesar y resultados obtenidos, y contemplan el envío por lotes (\emph{batch})
de datasets, el intercambio de \emph{ACKs} y señales de fin de flujo.

La implementación completa del protocolo se encuentra en:
\texttt{src/shared/communication\_protocol.py}.

\subsection{Mensajes}
\label{subsec:mensajes}

Como parte del protocolo, se estableció un conjunto de tipos de mensajes de
longitud fija (3 caracteres), orientados a cubrir las operaciones básicas del
sistema: envío de datasets por lotes, resultados de queries, reconocimientos
(\texttt{ACK}) y señalización de fin de flujo (\texttt{EOF}). Cada mensaje posee:

\begin{itemize}
  \item Un \textbf{tipo} de 3 caracteres (p.ej., \texttt{MIT}, \texttt{TRN}, \texttt{ACK}).
  \item Un \textbf{delimitador} de inicio de payload \texttt{[} y de fin \texttt{]}.
  \item Un \textbf{payload} cuyo formato depende del tipo (texto libre o lote estructurado).
\end{itemize}

A continuación se detalla el funcionamiento concreto, a partir del código provisto.

\subsection{Formato de los mensajes}
\label{subsec:formato}

\paragraph{Estructura general.}
Todo mensaje sigue un formato simple y siempre igual:

\begin{quote}
\texttt{<TYPE>[<PAYLOAD>]}
\end{quote}

\noindent Donde:
\begin{itemize}
  \item \texttt{<TYPE>} es un prefijo de \textbf{3 caracteres} (\texttt{MESSAGE\_TYPE\_LENGTH = 3}).
  \item El \textbf{payload} va entre corchetes: \verb|[| \ldots \verb|]|
        (\texttt{MSG\_START\_DELIMITER} y \texttt{MSG\_END\_DELIMITER}).
\end{itemize}

\paragraph{Payload por lotes (batch).}
Para los mensajes que transportan registros, el payload es un \emph{batch}:
\begin{quote}
\texttt{\char`\{}\texttt{\textless ROW\textgreater}\texttt{;}\texttt{\textless ROW\textgreater}\texttt{;}\texttt{...}\texttt{;}\texttt{\textless ROW\textgreater}\texttt{\char`\}}
\end{quote}

\noindent Cada \texttt{<ROW>} es una secuencia de pares clave--valor separados por comas:
\begin{quote}
\texttt{<FIELD>,<FIELD>,\ldots}
\end{quote}

\noindent Un \texttt{<FIELD>} tiene la forma:
\begin{quote}
\verb|{'<key>':'<value>'}|
\end{quote}

\newpage

\noindent Delimitadores usados en el batch:
\begin{itemize}
  \item Inicio/fin de batch: \texttt{BATCH\_START\_DELIMITER}=\verb|{|, \texttt{BATCH\_END\_DELIMITER}=\verb|}|
  \item Separador de filas: \texttt{BATCH\_ROW\_SEPARATOR}=\verb|;|
  \item Separador de campos: \texttt{ROW\_FIELD\_SEPARATOR}=\verb|,|
\end{itemize}

\noindent \textbf{Ejemplo de batch con dos filas:}
\begin{quote}
\verb|{|'id':'m001','name':'Latte','price':'4.50';'id':'m002','name':'Espresso','price':'3.20'\verb|}|
\end{quote}

Sin escape de caracteres: Las claves y valores no pueden contener comillas dobles,
dos puntos \texttt{:}, comas \texttt{,}, punto y coma \texttt{;}, ni llaves o corchetes.

\subsection{Tipos de mensajes y su propósito}
\label{subsec:tipos}

\begin{center}
\begin{tabular}{ll}
\hline
\textbf{Tipo (3 chars)} & \textbf{Uso} \\
\hline
\texttt{QRY} & Mensaje de consulta del cliente al sistema. \\
\texttt{ACK} & Reconocimiento genérico (éxito/recepción). \\
\texttt{MIT} & Lote de \emph{menu items}. \\
\texttt{STR} & Lote de \emph{stores}. \\
\texttt{TIT} & Lote de \emph{transaction items}. \\
\texttt{TRN} & Lote de \emph{transactions}. \\
\texttt{USR} & Lote de \emph{users}. \\
\texttt{Q1X} & Resultado para Query 1 (\emph{variant X}). \\
\texttt{Q21} & Resultado para Query 2.1. \\
\texttt{Q22} & Resultado para Query 2.2. \\
\texttt{Q3X} & Resultado para Query 3 (\emph{variant X}). \\
\texttt{Q4X} & Resultado para Query 4 (\emph{variant X}). \\
\texttt{EOF} & Señal de fin de flujo (payload: el tipo de flujo que finaliza). \\
\hline
\end{tabular}
\end{center}

Los tipos \texttt{MIT}, \texttt{STR}, \texttt{TIT}, \texttt{TRN} y \texttt{USR}
son \emph{batch messages}. Los tipos \texttt{Q1X}, \texttt{Q21}, \texttt{Q22},
\texttt{Q3X}, \texttt{Q4X} representan resultados de consultas y pueden usar el
mismo formato general de mensaje (el payload queda a criterio del productor del
resultado en este TP). \texttt{QRY} y \texttt{ACK} son mensajes de control
ligeros (texto breve o vacío).

\subsection{Codificación (\emph{encode})}
\label{subsec:encode}

\paragraph{Mensajes genéricos.}
La función privada \texttt{\_\_encode\_message(type, payload)} compone:
\[
\texttt{type} \parallel \texttt{[} \parallel \texttt{payload} \parallel \texttt{]}
\]
A partir de ella se exponen:

\begin{itemize}
  \item \texttt{encode\_ack\_message(msg: str) -> str}: Genera \texttt{ACK[msg]}.
  \item \texttt{encode\_eof\_message(message\_type: str) -> str}:
        Genera \texttt{EOF[<message\_type>]}, donde el \emph{payload} es el tipo
        de flujo que se declara finalizado (p.ej., \texttt{EOF[TRN]}).
\end{itemize}

\paragraph{Mensajes por lotes.}
\texttt{encode\_batch\_message(batch\_msg\_type, batch)}:
\begin{enumerate}
  \item Para cada \texttt{row: dict[str, str]}, codifica campos como
        \texttt{'key':'value'} unidos con \texttt{','}.
  \item Une filas con \texttt{';'} y encapsula con \texttt{'\{\}'}.
  \item Envuelve con el tipo y corchetes de mensaje.
\end{enumerate}
Se proveen \emph{helpers} tipados:
\texttt{encode\_menu\_items\_batch\_message},\texttt{encode\_stores\_batch\_message},
\texttt{encode\_transaction\_items\_batch\_message},
\texttt{encode\_transactions\_batch\_message},
\texttt{encode\_users\_batch\_message}.

\subsection{Decodificación (\emph{decode})}
\label{subsec:decode}

\paragraph{Acceso al payload y tipo.}
\begin{itemize}
  \item \texttt{decode\_message\_type(msg)}: Devuelve los primeros 3 caracteres.
        Valida largo mínimo; lanza \texttt{ValueError} si el mensaje es demasiado corto.
  \item \texttt{get\_message\_payload(msg)}: Remueve el prefijo de tipo y
        \texttt{[\,\,]} exteriores, retornando la cadena interna.
  \item \texttt{decode\_is\_empty\_message(msg)}: Verdadero si el payload está vacío.
\end{itemize}

\paragraph{Validación estructural.}
\texttt{\_\_assert\_message\_format(msg, expected\_type)} verifica:
\begin{enumerate}
  \item Que el tipo decodificado sea el esperado.
  \item Que el mensaje comience con \texttt{<expected>\,[} y termine con \texttt{]}.
\end{enumerate}
En caso contrario, lanza \texttt{ValueError}.

\paragraph{Decodificación de lotes.}
\texttt{decode\_batch\_message(msg)}:
\begin{enumerate}
  \item Obtiene el payload (que globalmente está entre \texttt{\{\}}).
  \item Separa filas por \texttt{;}.
  \item Para cada fila, \texttt{\_\_decode\_row}:
        \begin{enumerate}
          \item Remueve llaves \texttt{\{\}} en extremos (si las hubiera).
          \item Separa campos por \texttt{,}.
          \item Divide cada campo por el primer \texttt{:} en \texttt{key:value}.
          \item Quita comillas dobles exteriores de clave y valor.
        \end{enumerate}
  \item Devuelve \texttt{list[dict[str,str]]}.
\end{enumerate}
Se incluyen variantes tipadas que además validan el tipo del mensaje:
\texttt{decode\_menu\_items\_batch\_message}, \texttt{decode\_stores\_batch\_message},
\texttt{decode\_transaction\_items\_batch\_message},
\texttt{decode\_transactions\_batch\_message},
\texttt{decode\_users\_batch\_message}.

\paragraph{Señal de fin de flujo.}
\texttt{decode\_eof\_message(msg)} valida que el tipo sea \texttt{EOF} y retorna
el payload: el \emph{identificador de tipo de flujo} que finaliza (p.ej.,
\texttt{TRN}, \texttt{USR}, etc.).

\subsection{Ciclos de vida típicos de los mensajes}
\label{subsec:lifecycles}

\paragraph{Ingesta de datasets por lotes.}
Para cada dataset (p.ej., \texttt{MIT}, \texttt{TRN}):
\begin{enumerate}
  \item El productor envía uno o más mensajes \textbf{batch} del tipo correspondiente.
  \item Opcionalmente, el receptor responde con \texttt{ACK[\ldots]} para marcar recepción/éxito.
  \item Al finalizar el stream de ese dataset, el productor envía \texttt{EOF[<TIPO>]},
        por ejemplo \texttt{EOF[TRN]}.
\end{enumerate}

\paragraph{Consultas y resultados.}
\begin{enumerate}
  \item Un cliente emite \texttt{QRY[\ldots]} con parámetros de consulta (formato libre en este TP).
  \item Los resultados se retornan en mensajes \texttt{Q1X}, \texttt{Q21}, \texttt{Q22},
        \texttt{Q3X} o \texttt{Q4X}, según el ejercicio, con payload acorde.
  \item Se puede usar \texttt{ACK[\ldots]} para confirmar recepción o estado (p.ej., \emph{accepted}, \emph{done}).
\end{enumerate}

\subsection{Ejemplos de mensajes codificados}
\label{subsec:ejemplos}

\paragraph{ACK sin payload.}
\begin{verbatim}
ACK[]
\end{verbatim}

\paragraph{Batch de menu items (2 filas).}
\begin{verbatim}
MIT[{'id':'m001','name':'Latte','price':'4.50';'id':'m002','name':'Espresso','price':'3.20'}]
\end{verbatim}

\paragraph{Fin de flujo de transacciones.}
\begin{verbatim}
EOF[TRN]
\end{verbatim}

\paragraph{Resultado de Query 2.1 (payload libre en este TP).}
\begin{verbatim}
Q21[{'store_id':'s007','metric':'top_seller','value':'m001'}]
\end{verbatim}

\subsection{Validaciones, errores y robustez}
\label{subsec:validaciones}

\begin{itemize}
  \item \textbf{Tipo y formato:} Todo decodificador específico verifica que el
        tipo de mensaje coincida con el esperado y que existan los corchetes
        exteriores. Inconsistencias lanzan \texttt{ValueError}.
  \item \textbf{Longitud mínima:} \texttt{decode\_message\_type} exige al menos
        3 caracteres (tipo). Mensajes más cortos disparan \texttt{ValueError}.
  \item \textbf{Payload vacío:} Admitido (\texttt{ACK[]} o keep-alives).
  \item \textbf{Complejidad:} Las rutinas de encode/decode son lineales en la
        longitud del mensaje, con operaciones de \texttt{split}/\texttt{join}
        sobre separadores fijos, facilitando el procesamiento por streaming.
\end{itemize}

\subsection{Supuestos y limitaciones deliberadas}
\label{subsec:limitaciones}

\begin{itemize}
  \item \textbf{Sin escape de caracteres:} Claves y valores no deben contener
        comillas dobles \texttt{"}, dos puntos \texttt{:}, comas \texttt{,},
        punto y coma \texttt{;}, ni llaves/corchetes. Esto
        simplifica y acelera el parser a costa de restringir el dominio de
        valores posibles (suficiente para este TP).
  \item \textbf{Tipos de datos:} Todos los valores se tratan como cadenas
        (\texttt{str}); la tipificación semántica queda a cargo de las capas
        de negocio.
  \item \textbf{Orden y entrega:} El protocolo describe \emph{formato} y
        \emph{semántica de alto nivel} (batches, ACK, EOF). Las garantías de
        orden, reintentos o \emph{at-least-once/exactly-once} pertenecen a la
        capa de transporte y orquestación (p.ej., colas), tratadas en otra
        sección del informe.
\end{itemize}

\subsection{Extensibilidad}
\label{subsec:extensibilidad}

El diseño con prefijos de 3 caracteres facilita agregar nuevos tipos de
mensajes sin afectar a los existentes. Para incorporar uno nuevo basta con:

\begin{enumerate}
  \item Declarar la constante del tipo (3 letras).
  \item Implementar, si corresponde, \emph{helpers} de encode/decode análogos a los de batch.
  \item Documentar el payload asociado (texto libre o esquema de campos del batch).
\end{enumerate}

\newpage

\subsection{Resumen}
\label{subsec:resumen}

El protocolo define un \textbf{formato compacto y determinista}:
\begin{itemize}
\item encabezado de 3 caracteres para el tipo
\item delimitadores exteriores \texttt{[\,\,]}
\item para datasets, un \textbf{batch} con \texttt{\char`\{}\texttt{\char`\}} y separadores simples
\end{itemize}
Se proveen \emph{helpers} específicos para codificar/decodificar los distintos
datasets y una señal \texttt{EOF} que marca el fin de cada flujo lógico. Las
validaciones estructurales minimizan errores de parseo y el diseño favorece
procesamiento lineal y extensibilidad controlada para futuras necesidades del TP.

\newpage
\section{Mediciones de rendimiento del sistema implementado}

Para evaluar el rendimiento del sistema distribuido implementado, se realizaron diversas
mediciones bajo diferentes configuraciones y cargas de trabajo a lo largo de todo el desarrollo.

A continuación, se detallan los resultados obtenidos al finalizar el mismo, y su análisis correspondiente.

\subsection{Configuración del entorno de pruebas}

Las pruebas se llevaron a cabo en un entorno controlado, utilizando una red local con
múltiples nodos de cómputo.

Todos los nodos escalables, fueron configurados para levantar 2 nodos de cómputo cada uno.

Respecto a los nodos no escalables, solo se levantó un nodo de cómputo por cada uno (Es decir, para esta entrega no se implementó redundancia en los nodos no escalables, pero no se descarta para futuras versiones del desarrollo).

\subsection{Tiempo de procesamiento total}

Para cumplir con los requisitos brindados por la cátedra, donde se pedia que el sistema tarde menos de una hora en realizar
todo el procesamiento de las 4 consultas, con el dataset completo, se realizó la medición del tiempo total de procesamiento al finalizar
el desarrollo completo de la solución.

El tiempo total de procesamiento medido fue de exactamente '29' minutos con '32' segundos, cumpliendo así con el requisito establecido.

Con esto, se puede concluir que el sistema implementado es capaz de manejar eficientemente el procesamiento de las consultas
dentro del tiempo límite especificado, demostrando su efectividad y capacidad para trabajar con grandes volúmenes de datos
de manera distribuida.

}

\newpage
\justifying{
\section{Mecanismos de Control}

\subsection{Mecanismos de Control de Sincronización}

En el diseño del sistema distribuido cuenta con la incorporación de mecanismos de control 
de sincronización, con el objetivo de evitar problemas asociados a la concurrencia, tales 
como \textit{race conditions} o \textit{deadlocks}. 

Estos mecanismos garantizan un acceso ordenado a los recursos compartidos y la correcta 
coordinación entre los procesos en ejecución.  

\subsection{Mecanismos de Control de Señales y Finalización}

El sistema incorpora un manejo explícito de señales a fin de asegurar una finalización 
correcta y segura de los procesos.

En particular, se controla la señal \texttt{SIGTERM},
permitiendo un \textit{graceful quit} que libera recursos, cierra todos los archivos abiertos, se asegura de que todos los
file descriptors sean cerrados de forma correcta, mediante a los métodos de 'Close' y 'Delete' brindados por el Middleware.

Este mecanismo previene pérdidas de datos, mantiene la consistencia del sistema y asegura el correcto uso de los recursos del sistema operativo.

\subsection{Mecanismos de Control de Fallas}

Se contemplará el desarrollo de mecanismos de control de fallas para aumentar la robustez 
y confiabilidad del sistema. Estos mecanismos estarán orientados a gestionar situaciones 
adversas como caídas de procesos, pérdida de nodos de cómputo o interrupciones inesperadas 
durante la ejecución.  

El objetivo es garantizar, en la medida de lo posible, la continuidad del procesamiento 
y la recuperación parcial de información. En futuras versiones del documento se describirán 
las estrategias de detección, mitigación y recuperación ante fallas.

}

\newpage
\justifying{
\section{Desafíos al escalar el sistema distribuido}

Durante el proceso de escalabilidad del sistema distribuido, se identificaron diversos desafíos técnicos, particularmente en algunos nodos del flujo de procesamiento de datos. Las mayores dificultades se presentaron en los nodos \textit{reducers} (encargados de operaciones de agregación como \textit{sums} y \textit{counts}), en los nodos de ordenamiento (\textit{sorts}) y en los nodos de unión (\textit{joiners}).  
A continuación se detallan los principales aspectos abordados en cada caso.

\subsection{Reducers (Sums y Counts) y Sorts}

Los nodos \textit{reducers} y \textit{sorts} representan componentes \textbf{stateful}, es decir, mantienen un estado interno durante su ejecución. Este tipo de nodos plantea una mayor complejidad a la hora de diseñar estrategias de escalabilidad, ya que los datos no pueden simplemente distribuirse sin una política clara de particionado que preserve la coherencia del resultado.

A partir de la implementación de la entrega \textbf{MultiClient}, se logró una estrategia de escalabilidad efectiva basada en el \textbf{shardeo de los nodos} según una clave de partición lógica que permite dividir el cómputo sin afectar la consistencia de los resultados.

Por ejemplo, al calcular el \textit{Total Payment Value (TPV)} de una tienda específica, se puede aplicar un esquema de particionado por \texttt{store\_id}. De este modo, cada nodo \textit{reducer} procesa únicamente los datos correspondientes a un grupo de tiendas determinado, evitando interferencias y mejorando el rendimiento general del sistema.  
Esta metodología permitió escalar horizontalmente los nodos \textit{reducers} y \textit{sorts}, manteniendo la integridad de los datos y reduciendo el tiempo total de procesamiento en escenarios de alta concurrencia.

\subsection{Joiners}

Los nodos \textit{joiners} presentaron un desafío particular al trabajar en entornos con múltiples clientes, especialmente debido a la naturaleza dinámica del flujo de datos en \textit{streaming}. En este contexto, los nodos debían manejar simultáneamente tanto los datos que llegan en tiempo real como la información de nuevas fuentes o clientes que pueden incorporarse durante la ejecución.

Para resolver este problema, se diseñó una arquitectura basada en \textbf{hilos concurrentes} dentro del controlador principal, distribuidos de la siguiente manera:

\begin{enumerate}
    \item Un hilo dedicado a escuchar de forma continua el \textit{stream} de datos entrante, gestionando la transmisión y ejecutando las operaciones de \textit{join} cuando la información requerida está disponible.
    \item Un segundo hilo encargado de monitorear la llegada de nuevos conjuntos de datos provenientes de otras tablas o clientes, incorporándolos al flujo de unión en tiempo real.
    \item Un tercer hilo que actúa como \textbf{coordinador}, responsable de iniciar, supervisar y finalizar los otros dos hilos, garantizando una ejecución ordenada y un cierre seguro del sistema.
\end{enumerate}

Adicionalmente, se implementaron mecanismos de \textbf{bufferización y almacenamiento temporal} para retener los datos del \textit{stream} que aún no pueden ser procesados. Esto asegura que, cuando la información complementaria esté disponible, los datos pendientes puedan ser correctamente integrados en el \textit{join} correspondiente sin pérdida ni inconsistencia.

Esta arquitectura permitió alcanzar una ejecución más robusta y eficiente, garantizando la correcta sincronización de los flujos y la consistencia de los resultados, incluso bajo escenarios de concurrencia elevada y múltiples clientes activos.

}

\newpage
\justifying{
\section{Tests}
\label{sec:tests}

Con el objetivo de garantizar la correcta operación, consistencia y determinismo del
sistema distribuido desarrollado, se implementó un conjunto integral de pruebas
(\emph{tests}) que abordan distintos niveles de validación funcional y estructural.

En total, el sistema cuenta con \textbf{tres grandes grupos de tests}:

\begin{itemize}
    \item \textbf{Tests de Middleware}
    \item \textbf{Tests de Comparación de Outputs}
    \item \textbf{Tests de Propagación de EOF}
\end{itemize}

A continuación, se describen los objetivos, el alcance y los criterios de validación
de cada grupo de pruebas.

\subsection{Tests de Middleware}

Estos tests se encargan de verificar la \textbf{correcta implementación y funcionamiento
del Middleware} desarrollado sobre la biblioteca de comunicación \texttt{RabbitMQ},
siguiendo la interfaz y las especificaciones brindadas por la cátedra.

El objetivo principal de este conjunto de pruebas es asegurar que la capa de comunicación
entre nodos funcione de manera estable, confiable y conforme al protocolo definido,
tanto para el envío como para la recepción de mensajes.  
Entre los aspectos validados se incluyen:

\begin{itemize}
    \item Correcta creación y vinculación de colas y \emph{exchanges}.
    \item Enrutamiento adecuado de mensajes según el tipo de flujo o dataset.
    \item Manejo de \emph{acknowledgements} y confirmaciones de entrega.
    \item Comportamiento esperado del middleware bajo condiciones de carga y escalado.
    \item Gestión de excepciones.
\end{itemize}

Estos tests fueron desarrollados originalmente para la entrega
\textbf{``Escalabilidad, Middleware y Coordinación de Procesos''}, y sirvieron de base
para validar la estabilidad de la infraestructura de mensajería del sistema.

\subsection{Tests de Comparación de Outputs}

El segundo conjunto de pruebas está orientado a la \textbf{validación funcional de los resultados}
obtenidos por el sistema distribuido, comparando las salidas producidas por las consultas
con los resultados esperados.

Dado que en un entorno distribuido las tareas se dividen entre nodos escalables sin
garantizar un orden específico en la entrega de resultados, fue necesario diseñar
técnicas de comparación \textbf{insensibles al orden} para validar el correcto
comportamiento determinístico del sistema.  
De esta forma, las pruebas se centran en comprobar la equivalencia semántica
de los resultados más allá del orden particular de las filas.

\newpage

\paragraph{Validación por consulta.}

\begin{itemize}
    \item \textbf{Query 1:}  
    Verifica que la cantidad de líneas generadas sea exactamente la misma que la esperada,
    confirmando que el volumen de resultados es correcto.

    \item \textbf{Query 2:}  
    Compara tanto la cantidad de filas como el contenido de cada una,
    sin considerar el orden en que fueron entregadas.  
    De este modo, se valida que los resultados sean completos y correctos, independientemente
    del orden de procesamiento distribuido.

    \item \textbf{Query 3:}  
    Aplica los mismos criterios que en la Query 2, validando que las filas generadas
    contengan la misma información y que no haya pérdidas ni duplicados,
    aún cuando el orden de emisión varíe entre ejecuciones.

    \item \textbf{Query 4:}  
    Verifica que:
    \begin{itemize}
        \item La cantidad total de filas coincida con la esperada.
        \item Exista la misma cantidad de respuestas por cada tienda.
        \item Los clientes identificados como los más compradores por tienda
              coincidan en cantidad de compras.
    \end{itemize}
    No se consideran las diferencias en nombres o fechas de nacimiento, ya que
    en caso de empates el orden de los tres máximos puede variar entre ejecuciones.
    Lo relevante es que los valores reportados correspondan a los tres mayores por tienda,
    garantizando así la corrección lógica de la consulta.
\end{itemize}

Estos tests fueron desarrollados inicialmente para la entrega
\textbf{``Escalabilidad, Middleware y Coordinación de Procesos''} y luego
\textbf{mejorados para la entrega ``MultiClient''}.

\subsection{Tests de Propagación de EOF}

Finalmente, se implementó un conjunto de pruebas específicas para validar el
\textbf{mecanismo de propagación de mensajes \texttt{EOF}} dentro del sistema
\emph{MultiClient}.

El objetivo de estos tests es comprobar que:
\begin{itemize}
    \item Cada controlador reciba correctamente los \texttt{EOF} asociados a los flujos
          que le corresponden, diferenciados por \texttt{session\_id}.
    \item La propagación de \texttt{EOF} se realice de forma completa y ordenada hacia
          los nodos sucesores en la cadena de procesamiento.
    \item No existan pérdidas, duplicados ni propagaciones cruzadas entre sesiones.
\end{itemize}

De esta manera, se valida que el mecanismo de cierre de flujos por sesión funcione
correctamente y que el sistema distribuido finalice el procesamiento de cada cliente
de forma independiente y controlada, sin afectar el estado de los demás flujos activos.

Estos tests fueron desarrollados como parte de la entrega
\textbf{``MultiClient''}, a pedido de la cátedra, para asegurar la correcta
implementación del sistema de detección y propagación de fin de flujo dentro del
entorno multiusuario.

}

\newpage
\justifying{
\section{Tolerancia a Fallas}

\subsection{Introducción}

En esta sección se detallan los mecanismos implementados en el Sistema Distribuido para tolerar distintos escenarios de falla.  
El objetivo principal de estas implementaciones es garantizar la \textbf{consistencia} y la \textbf{robustez} del sistema, priorizando estos atributos por sobre la eficiencia en términos de rendimiento.

En cuanto a la arquitectura general, no se presentan modificaciones en la topología previamente documentada.  
El único agregado corresponde a la incorporación de un conjunto de nodos denominados \textit{Health Checkers}, encargados de consultar de forma recurrente el estado de cada nodo del sistema, con el fin de validar su correcto funcionamiento o, en caso de detectar fallas, iniciar procesos de recuperación para restablecer las instancias caídas.

\subsection{Health Checkers}

Los nodos \textit{Health Checkers} se encargan de monitorear continuamente que todos los nodos encargados del procesamiento para la generación de resultados se encuentren activos.  
Este mecanismo está implementado mediante múltiples instancias dispuestas en una topología en anillo, con el fin de garantizar la tolerancia a fallas y la alta disponibilidad del propio sistema de monitoreo.

Uno de los nodos cumple el rol de \textbf{Líder}, siendo el responsable de coordinar las validaciones periódicas.  
En caso de que el líder falle, se ejecuta automáticamente un algoritmo de elección que designa un nuevo líder, el cual retoma las tareas del anterior.

El proceso de verificación consiste en el envío periódico de mensajes de tipo \textit{Ping} a cada nodo del sistema.  
Si el nodo responde con un \textit{ACK}, se considera activo; en caso contrario, el líder inicia el proceso de recuperación de esa instancia.

\subsubsection{Topología del Sistema de Health Checkers}

A continuación se deja un diagrama que representa la topología del \textbf{Sistema de Health Checkers}, donde se representa tanto la lógica de
topología de anillo, como la función del \textbf{Líder} y un ejemplo de funcionamiento al hacer \textit{Ping} a diversos \textit{Cleaners}.

\vspace{0.5cm}

\FloatBarrier
\begin{figure}[H]
  \centering
  \includegraphics[scale=0.22, keepaspectratio]{health.png}
  \caption{Topología del Sistema de Health Checkers}
  \label{fig:healthcheckers}
\end{figure}

\newpage

\subsection{Cambio de comportamiento en los nodos}

Se introdujeron modificaciones significativas en el funcionamiento de todos los nodos del sistema que consumen mensajes desde las colas de \textit{RabbitMQ}.  
El objetivo de estos cambios es fortalecer la robustez del sistema ante fallas parciales y garantizar la consistencia de los datos en todo momento.

Cada nodo, al recibir (``popear'') un mensaje desde la cola correspondiente, ejecuta su lógica de procesamiento, le envía el mensaje a la siguiente cola de el próximo nodo, escribe el mensaje o el estado actualizado en su volumen persistente, y finalmente envía el \textit{ACK} a \textit{RabbitMQ}.  
Este orden de operaciones —procesamiento, envío, persistencia y confirmación— es esencial para evitar duplicaciones o pérdidas de mensajes ante fallas inesperadas.  

Gracias a la semántica de confirmación de \textit{RabbitMQ}, si un mensaje es consumido pero no se envía el \textit{ACK}, dicho mensaje se reencola automáticamente al inicio de la cola.  
De esta forma, ante una caída, el nodo podrá retomar su ejecución desde el último estado persistido o reprocesar el mensaje que quedó pendiente, garantizando así la consistencia del flujo.

\textbf{Observación:}  
Todos los archivos generados se almacenan en volúmenes persistentes asociados a los contenedores de cada nodo.  
Esto permite que, incluso tras una caída o reinicio, el nodo conserve su información y pueda recuperar su estado previo sin pérdida de datos.

\subsubsection{Nodos Stateless}

Los nodos \textit{Stateless} procesan mensajes de manera independiente, sin mantener un estado acumulativo entre ellos.  
Su flujo de ejecución se compone de las siguientes etapas:

\begin{enumerate}
    \item Recepción del mensaje desde la cola de entrada.
    \item Procesamiento del mensaje (limpieza, filtrado u otra operación específica).
    \item Envío del resultado al siguiente nodo.
    \item Escritura del mensaje procesado en un archivo local del volumen persistente, para disponer de una copia de respaldo.
    \item Envío del \textit{ACK} a \textit{RabbitMQ}.
\end{enumerate}

La escritura en disco se realiza \textbf{únicamente al final del proceso}, luego de haber enviado el mensaje.  
Este orden es crítico, ya que evita que un mensaje pueda ser reenviado o computado más de una vez, garantizando la consistencia global del sistema.

\subsubsection{Nodos Stateful (Joins)}

Los nodos \textit{Joins} presentan un comportamiento intermedio entre los nodos con y sin estado.  
Durante su ejecución inicial, reciben información de configuración proveniente del servidor, la cual se almacena en su volumen local y se utiliza de manera incremental al procesar los siguientes mensajes.

Una vez recibida toda la información del servidor, el nodo pasa a comportarse de forma análoga a un nodo \textit{Stateless}:  
procesa, envía, registra y confirma cada mensaje de manera independiente.

Para facilitar la recuperación ante fallas, una vez completada la fase de inicialización, el nodo \textit{Join} registra un indicador distintivo (por ejemplo, una marca de finalización) en su archivo local.  
De esta forma, si el nodo se reinicia, puede detectar fácilmente si ya recibió todos los datos del servidor o si aún resta información por almacenar.

\newpage

\subsubsection{Nodos Stateful con estado acumulativo (Sorts, Sums, Counts)}

\textbf{Funcionamiento inicial del nodo}

Los nodos con estado acumulativo —como \textit{Sorts}, \textit{Sums} y \textit{Counts}— mantienen información agregada entre mensajes.  
Su comportamiento fue rediseñado para maximizar la confiabilidad y optimizar la recuperación ante fallas.

Cada nodo sigue el siguiente flujo:

\begin{enumerate}
    \item Recibe un mensaje desde la cola correspondiente.
    \item Actualiza su estado interno con la información del mensaje (operación de cómputo o agregación).
    \item Escribe el mensaje procesado en su archivo local (uno por cada nodo antecesor).
    \item Envía el \textit{ACK} a \textit{RabbitMQ}.
    \item Continúa con el siguiente mensaje.
\end{enumerate}

Cada cierto número de mensajes $n$, configurable según la carga del nodo o la probabilidad de falla estimada, se realiza un \textbf{backup o checkpoint} del estado actual.  
Este se guarda en un archivo separado dentro del mismo volumen, y actúa como punto de restauración ante fallas.

\textbf{Envío del reporte generado hacia el próximo nodo}

Finalmente, una vez completado el procesamiento de todos los mensajes correspondientes a una consulta o cliente, el nodo realiza el cálculo del estado final acumulado.  
Antes de enviarlo al siguiente nodo de la secuencia, dicho estado se escribe en el archivo de \textbf{backup/checkpoint}, garantizando la persistencia del resultado más reciente.

Posteriormente, se procede al envío del estado al nodo siguiente.  
Una vez completado este envío, el nodo registra en su volumen una marca o indicador que señala que el estado correspondiente a esa consulta o cliente ya fue transmitido correctamente.  
Esto permite que, en caso de una caída y posterior recuperación, el nodo identifique que el resultado ya fue enviado y evite duplicaciones o recomputaciones innecesarias.

En los casos en que el envío del estado final requiera dividir la información en múltiples mensajes, se aplica la misma lógica de procesamiento y persistencia que para el resto de los mensajes:  
cada fragmento enviado se escribe previamente en un volumen dedicado al manejo de dichos estados parciales, asegurando un control preciso de la consistencia y la trazabilidad del proceso completo.

\textbf{Consideraciones}

Para mantener el uso eficiente del almacenamiento, tras cada backup se eliminan o marcan los registros previos como \textit{COMMIT}, indicando que los mensajes anteriores ya fueron computados y persistidos en el estado actual.

Cada nodo mantiene archivos separados por cada uno de sus antecesores.  
Esto permite acceder de manera directa al último mensaje procesado por cada fuente, manteniendo la trazabilidad y la consistencia del flujo completo.  
Se asume además que todas las escrituras en disco son \textbf{atómicas}, garantizando que sólo existen dos posibles resultados: escritura completa o ausencia total de la misma, eliminando la posibilidad de estados intermedios inconsistentes.

\newpage

\subsection{Cambios en el protocolo}

La nueva implementación introduce un encabezado (\textit{header}) ligero que se agrega sobre el paquete existente, siguiendo un enfoque análogo al apilado de cabeceras entre capas de enlace y red en redes tradicionales. Este encabezado incorpora dos campos adicionales:

\begin{itemize}
  \item \textbf{UUID del mensaje} (\texttt{msg\_uuid}): Generado por el servidor emisor inicial y \textit{consistente durante toda la vida del mensaje} a través de la tubería de procesamiento. En los nodos que generan reportes/outputs finales, se genera un \textit{nuevo} UUID para el artefacto resultante, el cual se usa a partir de ahí para su trazabilidad.
  \item \textbf{Identificador del nodo origen} (\texttt{src\_node\_id}): Indica el nodo que produjo el mensaje. Se utiliza en el receptor para dirigir la persistencia al volumen/archivo correspondiente a ese origen, preservando trazabilidad y consistencia.
\end{itemize}

\FloatBarrier
\begin{figure}[H]
  \centering
  \includegraphics[scale=0.18, keepaspectratio]{mensajes.png}
  \caption{Estructura de los mensajes}
  \label{fig:estructura_mensajes}
\end{figure}

\paragraph{Balanceo determinista basado en hash}
Los nodos que antes distribuían carga con \textit{Round Robin} (contador) migran a \textbf{hash por \texttt{msg\_uuid}}. Este cambio evita desbalances e inconsistencias cuando un nodo cae y reinicia su contador:
\begin{itemize}
  \item Con \textit{Round Robin}, un reinicio cambia el punto del ciclo y el mismo mensaje podría enrutarse a un destino distinto.
  \item Con \textbf{hash(msg\_uuid)}, a igual UUID se obtiene el mismo destino (consistentemente), incluso si el nodo se reinicia. Esto preserva afinidad de mensaje y localización esperada del estado.
\end{itemize}

\paragraph{Compatibilidad y consistencia}
El encabezado es aditivo, no rompe el formato del payload previo. La escritura local sigue siendo atómica, y la lógica de ACK de \textit{RabbitMQ} permanece inalterada: el ACK sólo se emite cuando el procesamiento y la persistencia asociada se completan, manteniendo idempotencia extremo a extremo.

\newpage

\subsection{Escenarios de Falla}

A continuación se detallan los distintos escenarios de falla contemplados durante el diseño de los mecanismos de tolerancia y recuperación del sistema.  
En todos los casos, se garantiza tanto la continuidad operativa ante la caída de uno o más nodos, como la consistencia de los resultados y del flujo de mensajes.

\subsubsection{Falla en los Health Checkers}

Ante la caída de algún nodo del sistema de \textit{Health Checkers}, pueden presentarse dos situaciones:

\begin{itemize}
    \item \textbf{Falla del líder:}  
    Se ejecuta un proceso de elección para designar un nuevo líder, el cual se reconecta con los nodos activos del sistema y retoma las tareas pendientes.

    \item \textbf{Falla de un nodo no líder:}  
    La instancia se remueve temporalmente del anillo para mantener la coherencia de la topología.
\end{itemize}

En ambos casos, se intenta levantar nuevamente la instancia caída para restablecer el nivel de redundancia definido y mantener la alta disponibilidad del sistema.

\subsubsection{Falla en los Nodos Stateless y Nodos Join}

El comportamiento ante fallas es similar al de los nodos \textit{Join}:

\begin{itemize}
    \item \textbf{Falla antes del \textit{ACK}:}  
    \textit{RabbitMQ} reencola el mensaje, que será procesado nuevamente cuando el nodo se recupere.
    Como esta es la última operación antes de popear otro mensaje, no existe otro instante de falla a considerar.
\end{itemize}

\subsubsection{Falla en los Nodos Statefull}

Dependiendo del tipo de nodo, los mecanismos de recuperación difieren:

\paragraph{Nodos con estado acumulativo (Sorts, Sums, Counts):}
\begin{itemize}
    \item \textbf{Si el nodo falla antes de procesar el mensaje}:  
    El mensaje se reencola automáticamente por \textit{RabbitMQ}, y al reiniciarse, el nodo lo vuelve a procesar desde su archivo o desde el backup más reciente.

    \item \textbf{Si falla después de procesar el mensaje pero antes del \textit{ACK}}:  
    El mensaje se reencola, y al recuperarse, el nodo detecta en su registro que el mensaje ya fue computado, evitando duplicaciones.

    \item \textbf{Si falla durante la acumulación}:  
    Al reiniciarse, el nodo restaura su estado desde el último backup disponible y continúa procesando desde el siguiente mensaje.

    \item \textbf{Si falla en el proceso de envío del reporte hacia el próximo nodo}:  
    Al volver a despertar, se retoma desde el último mensaje enviado, para garantizar la consistencia del sistema, evitando mensajes duplicados.
    Para identificar los mensajes, el nodo que genera el reporte asigna los IDs correspondientes, haciendose responsable de su correctitud.
\end{itemize}

\newpage

\subsubsection{Fallas Desestimadas}

Durante el análisis se identificaron otros posibles tipos de falla:

\begin{itemize}
    \item Falla en el \textbf{Servidor Central}.
    \item Falla en los \textbf{Clientes}.
    \item Falla en el \textbf{Middleware RabbitMQ}.
\end{itemize}

El manejo de estos escenarios fue desestimado en esta versión del sistema, dado que implicaría definir contratos adicionales con los clientes y un nivel de complejidad que excede los requerimientos del presente enunciado.  

Sin embargo, se documentan estos casos para su consideración en futuras versiones del sistema.

\subsubsection{Mecanismos adicionales de consistencia}

Para garantizar la coherencia incluso en escenarios de fallas simultáneas:
\begin{itemize}
    \item El nodo siguiente a uno caído valida que los mensajes recibidos no sean duplicados, comparando el último mensaje recibido con el nuevo.
    \item Los nodos que mantienen estado lo recomponen desde su último backup; los que no, reprocesan los mensajes pendientes desde la cola.
    \item Dado que \textit{RabbitMQ} no pierde mensajes durante las caídas, el sistema mantiene su integridad global.
\end{itemize}

\subsection{Sistema de Generación de Errores}

Con el objetivo de validar los mecanismos de tolerancia a fallas implementados en el sistema, se desarrolló un \textbf{Sistema de Generación de Errores}, diseñado para simular distintos escenarios de falla de manera controlada o aleatoria.

Este sistema cuenta con dos funcionalidades principales:

\begin{itemize}
    \item \textbf{Inyección manual de fallas en instancias específicas:}  
    Esta herramienta permite provocar la caída de instancias puntuales del sistema de forma controlada.  
    Su propósito principal es facilitar las pruebas durante el desarrollo, permitiendo analizar el comportamiento del sistema ante la falla y posterior recuperación de nodos concretos, así como verificar la consistencia de los datos y estados luego de cada reinicio.

    \item \textbf{Módulo de fallas aleatorias tipo \textit{Chaos Monkey}:}  
    Inspirado en las prácticas de ingeniería del caos, este módulo genera caídas y perturbaciones de manera aleatoria en distintas partes del sistema.  
    De esta forma, se evalúa el comportamiento integral del sistema distribuido bajo condiciones de falla no determinísticas, más cercanas a un entorno real de producción.  
    Este enfoque permite demostrar la robustez global y la correcta recuperación de los componentes frente a fallas imprevistas.
\end{itemize}

La señal que envía el sistema de generación de fallas a los nodos vivos es \textbf{SIGKILL}. 

En conjunto, ambas funcionalidades permiten validar tanto el funcionamiento individual de los mecanismos de tolerancia a fallas como su efectividad a nivel global, asegurando la resiliencia y estabilidad del sistema frente a distintos tipos de incidentes.

\newpage

\subsection{Mediciones de Rendimiento (Con Fallas)}

Con el objetivo de evaluar el comportamiento y la eficiencia del sistema distribuido, se realizaron diversas mediciones de rendimiento bajo distintos escenarios y volúmenes de datos.  
El propósito de este análisis es determinar la capacidad del sistema para mantener la consistencia de los resultados y la estabilidad de los tiempos de respuesta ante la presencia de fallas controladas o aleatorias.

Las pruebas se dividen en dos grupos principales: aquellas realizadas sobre un \textbf{dataset reducido}, orientadas a validar la consistencia frente a fallas aleatorias, y aquellas efectuadas sobre el \textbf{dataset completo}, destinadas a medir la performance global del sistema con y sin eventos de falla.

\subsubsection{Dataset Reducido: Consistencia frente a caídas aleatorias}

En este conjunto de pruebas se busca demostrar que el sistema mantiene resultados consistentes ante la ocurrencia de fallas aleatorias, validando así la efectividad de los mecanismos de tolerancia a fallas implementados.

\paragraph{Herramienta de evaluación}
Para la ejecución de estas pruebas se empleará el \textbf{Sistema de Generación de Errores} previamente descripto, en su modo de funcionamiento aleatorio tipo \textit{Chaos Monkey}.  
Esta herramienta permitirá simular caídas no determinísticas de distintas instancias del sistema, abarcando tanto nodos \textit{stateless} como \textit{stateful} y componentes del anillo de \textit{Health Checkers}.

\paragraph{Parámetros de prueba}
Los principales parámetros de configuración de las pruebas son los siguientes (valores a definir en futuras iteraciones):

\begin{itemize}
    \item Frecuencia de generación de fallas: \texttt{X segundos}.
    \item Duración total de la prueba: \texttt{X minutos}.
    \item Cantidad de nodos involucrados: \texttt{X}.
    \item Tipos de nodos afectados: \texttt{Stateful / Stateless / Health Checker}.
    \item Volumen de datos procesado: \texttt{X Gb}.
\end{itemize}

\paragraph{Criterios de evaluación}
Los criterios utilizados para validar la consistencia y estabilidad del sistema serán:

\begin{itemize}
    \item Coincidencia de los resultados obtenidos con los esperados (control de consistencia).
    \item Tiempo promedio de recuperación ante fallas.
    \item Número de tareas reintentadas correctamente.
    \item Impacto en la latencia promedio del sistema.
\end{itemize}

\paragraph{Resultados esperados}
Se espera observar resultados consistentes en la salida final, independientemente de las caídas aleatorias introducidas, y una recuperación rápida de las instancias afectadas.  
Los valores numéricos y métricas específicas serán completados una vez finalizada la ejecución de las pruebas.

\newpage

\subsubsection{Dataset Completo: Monitoreo de performance con y sin fallas}

Este conjunto de pruebas se orienta a analizar el rendimiento del sistema distribuido bajo carga completa, tanto en condiciones estables (sin fallas) como durante la inyección de fallas aleatorias, con el fin de evaluar la degradación del desempeño.

\paragraph{Herramienta y entorno de medición}
Las mediciones se realizarán utilizando herramientas de monitoreo de recursos y trazabilidad, tales como \texttt{Prometheus}, \texttt{Grafana} o equivalentes (a definir).  
Estas permitirán recolectar métricas en tiempo real sobre el consumo de CPU, memoria, tráfico de red y latencia promedio de las consultas distribuidas.

\paragraph{Parámetros de prueba}
Los principales parámetros definidos para este escenario son los siguientes (valores a completar):

\begin{itemize}
    \item Tamaño del dataset completo: \texttt{X Gb}.
    \item Cantidad total de nodos en ejecución: \texttt{X}.
    \item Cantidad de consultas simultáneas: \texttt{X}.
    \item Duración de la medición: \texttt{X minutos}.
    \item Inyección de fallas aleatorias: \texttt{Sí / No}.
\end{itemize}

\paragraph{Criterios de evaluación}
Las métricas principales a analizar serán:

\begin{itemize}
    \item Tiempo promedio de procesamiento por consulta.
    \item Throughput del sistema (consultas por segundo).
    \item Utilización promedio de recursos (CPU, memoria, red).
    \item Variación del rendimiento ante la presencia de fallas.
\end{itemize}

\paragraph{Resultados esperados}
Se espera observar un desempeño estable del sistema bajo carga completa en condiciones normales, con una degradación controlada durante las pruebas con fallas aleatorias.  
Los resultados cuantitativos y gráficos comparativos serán incorporados una vez completadas las mediciones experimentales.

}

\newpage
\justifying{
\section{Cronograma teórico del desarrollo}

\subsection{Diseño}

Previo al inicio del desarrollo efectivo del diseño, el equipo realizó una 
instancia de planificación para organizar la distribución de responsabilidades.  
El objetivo fue asegurar que cada integrante asumiera un conjunto de tareas equilibrado, 
alineado tanto con sus fortalezas como con las necesidades del proyecto.  

La planificación se estructuró en función de las vistas arquitectónicas requeridas, 
los diagramas de modelado, la documentación conceptual y las definiciones de base.

\subsection{Escalabilidad}

El desarrollo de esta entrega cuenta con un tiempo estimado de 2.5 semanas.  
Durante la primera semana se realizará un análisis detallado de los criterios de escalabilidad, 
los requisitos técnicos y el alcance de la entrega, así como la planificación y distribución de tareas 
entre los integrantes del equipo.  

En las semanas posteriores se definirán y ejecutarán las labores específicas de cada integrante, 
las cuales se documentarán en versiones futuras de este informe.  
La presente sección cumple únicamente la función de establecer la planificación inicial.

\subsection{Multi-Client}

El desarrollo de esta entrega cuenta con un tiempo estimado de 2 semanas.  
La primera semana estará destinada al análisis de los criterios, requisitos y alcance de la 
funcionalidad de múltiples clientes, junto con la planificación y asignación de tareas a cada 
integrante del equipo.  

En la semana restante se abordará la ejecución de las labores planificadas, que se detallarán 
en próximas versiones de este documento.  
En esta instancia se deja asentada únicamente la planificación teórica.

\subsection{Tolerancia}

El desarrollo de esta entrega cuenta con un tiempo estimado de 3.5 semanas.  
La primera semana se dedicará al análisis de criterios, requisitos de tolerancia a fallos y alcance 
de la entrega, además de la planificación y distribución de responsabilidades entre los integrantes.  

Las semanas siguientes estarán orientadas a la implementación progresiva de los mecanismos 
planificados.  
Las labores específicas por integrante se incluirán en futuras actualizaciones de este documento.  

\subsection{Paper}

El desarrollo de esta entrega cuenta con un tiempo estimado de 1.5 semanas.  
Durante la primera semana se llevará a cabo el análisis de criterios, requisitos de formato y 
alcance de la entrega, así como la planificación y reparto de tareas.  

El tiempo restante será utilizado para la redacción y consolidación del documento final, 
cuyos detalles se incorporarán en versiones posteriores de este informe.  
La presente sección constituye únicamente la documentación inicial de la planificación.

\newpage
\section{Cronograma real del desarrollo}

\subsection{Diseño}

El desarrollo del diseño se llevó a cabo en un período de dos semanas, 
conformando un equipo de tres integrantes.  

Durante la \textbf{primera semana}, el grupo se enfocó en interpretar en profundidad 
los requisitos planteados en el enunciado, analizar la lógica de las consultas y 
definir el diseño conceptual del sistema. Asimismo, en esta etapa se realizó la 
repartición de tareas entre los integrantes, de manera de optimizar el tiempo de 
ejecución restante.

En la \textbf{segunda semana}, cada integrante se abocó a las actividades asignadas, 
según el siguiente detalle:

\subsection*{Luciano}
\begin{itemize}
    \item Vista Lógica.
    \item Vista de Desarrollo.
    \item Vista Física.
    \item Vista de Procesos.
    \item Diseño inicial/conceptual del Middleware.
    \item Diagrama de Paquetes.
\end{itemize}

\subsection*{Axel}
\begin{itemize}
    \item Escenarios de uso (Casos de Uso).
    \item Diagrama de Secuencia.
    \item Diagrama de Actividades.
    \item Diagrama de Robustez.
    \item DAG (Directed Acyclic Graph).
\end{itemize}

\subsection*{Felipe}
\begin{itemize}
    \item Redacción de las definiciones iniciales.
    \item Creación de los cronogramas.
    \item Descripción detallada de la ejecución de las consultas.
    \item Explicaciones introductorias de los mecanismos de control y protocolos.
\end{itemize}

\newpage

\subsection{Middleware y Coordinación de Procesos}

El desarrollo de la primera versión del sistema distribuido se llevó a cabo en un período de dos semanas y media, 
conformando un equipo de tres integrantes.  

Durante la \textbf{primera semana}, el grupo se enfocó en interpretar en profundidad 
los requisitos planteados en el enunciado, analizar posibles implementaciones para las consultas solicitadas,
definir el lenguaje de programación a utilizar, pactar protocolos e identificar bibliotecas necesarias para el desarrollo.

Asimismo, en esta etapa se realizó la repartición de tareas entre los integrantes, de manera de optimizar el tiempo de 
ejecución restante.

En la \textbf{segunda semana}, cada integrante se abocó a las actividades asignadas, 
según el siguiente detalle:

\subsection*{Luciano}
\begin{itemize}
    \item Implementación del Middleware.
    \item Creación de los tests unitarios para validad funcionamiento del Middleware.
    \item Establecimiento de conexiones entre Cliente y Servidor.
    \item Redacción de protocolos de comunicación.
    \item Integración del Middleware con los controladores.
\end{itemize}

\subsection*{Axel}
\begin{itemize}
    \item Implementación de los controladores:
    \begin{itemize}
        \item Join.
        \item Sorts.
        \item Maps.
        \item Filters.
    \end{itemize}
    \item Integración de los controladores con el Middleware.
    \item Construcción del ecosistema de nodos de cómputo distribuido mediante a Dockerfiles.
    \item Medición del rendimiento del sistema.
\end{itemize}

\subsection*{Felipe}
\begin{itemize}
    \item Implementación de los controladores:
    \begin{itemize}
        \item Cleaners.
        \item Output Builders.
        \item Reduces.
    \end{itemize}
    \item Corrección del diseño inicial según las necesidades del desarrollo y las correcciones indicadas por el docente a cargo.
    \item Redacción de la nueva documentación del sistema.
    \item Implementación de los mecanismos de cierre 'graceful' de los nodos de cómputo.
\end{itemize}

}

\newpage
\end{document}