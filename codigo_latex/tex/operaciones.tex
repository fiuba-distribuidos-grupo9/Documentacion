\section{Definición Técnica del Sistema Distribuido}

El sistema distribuido propuesto permite al usuario interactuar mediante una consola, 
desde la cual los usuarios envían las instrucciones y datasets necesarios para el procesamiento, y a cambio reciben como salida de la misma el resultado de las consultas realizadas. 
La infraestructura se implementará sobre contenedores Docker, los cuales emulan 
múltiples nodos computacionales. Cada nodo se especializa en la ejecución de una operación 
determinada, tales como filtrado, mapeo, reducción, ordenamiento o combinación de datos.  

El sistema recibe como entrada un conjunto de archivos con la información a procesar:

\begin{itemize}
    \item \texttt{menu\_items.csv}
    \item \texttt{payment\_methods.csv}
    \item \texttt{stores.csv}
    \item Múltiples archivos \texttt{transaction\_items.csv}
    \item Múltiples archivos \texttt{transactions.csv}
    \item Múltiples archivos \texttt{users.csv}
    \item \texttt{vouchers.csv}
\end{itemize}

Estos archivos son distribuidos entre los distintos nodos según lo que requiera cada operación. 
La comunicación entre el nodo coordinador y los nodos de procesamiento se gestiona mediante 
un \textit{middleware}, el cual se describe en detalle en una sección posterior.

\subsection{Operaciones Distribuidas}

Las operaciones fundamentales soportadas por el sistema son las siguientes:

\begin{enumerate}
    \item \textbf{Filter}: Permite seleccionar subconjuntos de datos en función de condiciones 
    específicas. Una misma instancia puede aplicar diferentes filtros según el criterio 
    definido en la consulta (por ejemplo, filtrar por fechas, montos o rangos horarios).
    
    \item \textbf{Map}: Transforma los datos de entrada generando nuevas columnas o 
    reformateando la información existente. Ejemplo: Agregar un campo con el mes y año a partir 
    de una fecha, o calcular un contador auxiliar.
    
    \item \textbf{Reduce}: Agrupa datos por una clave determinada y aplica una función de 
    agregación (como sumas, conteos o acumulaciones). Existen múltiples variantes, dependiendo 
    de qué métrica se busque consolidar.
    
    \item \textbf{Join}: Combina registros de diferentes datasets en función de un campo común, 
    permitiendo enriquecer la información (ejemplo: Unir transacciones con usuarios para obtener 
    la fecha de nacimiento del cliente).
    
    \item \textbf{SortBy}: Ordena los registros de acuerdo con uno o más criterios, ya sea de 
    manera ascendente o descendente. Es fundamental para identificar los elementos más 
    significativos en cada consulta (por ejemplo, productos más vendidos).
\end{enumerate}

Cada nodo de operación está diseñado para ser genérico y flexible: Puede ejecutar cualquier variante de las funciones mencionadas según el rol específico del nodo, siendo el sistema coordinador quien le envía los datos y la 
configuración necesaria para que realice la tarea solicitada.