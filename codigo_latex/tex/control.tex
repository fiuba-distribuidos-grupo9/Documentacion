\section{Mecanismos de Control}

\subsection{Mecanismos de Control de Sincronización}

En el diseño del sistema distribuido cuenta con la incorporación de mecanismos de control 
de sincronización, con el objetivo de evitar problemas asociados a la concurrencia, tales 
como \textit{race conditions} o \textit{deadlocks}. 

Estos mecanismos garantizan un acceso ordenado a los recursos compartidos y la correcta 
coordinación entre los procesos en ejecución.  

\subsection{Mecanismos de Control de Señales y Finalización}

El sistema incorpora un manejo explícito de señales a fin de asegurar una finalización 
correcta y segura de los procesos.

En particular, se controla la señal \texttt{SIGTERM},
permitiendo un \textit{graceful quit} que libera recursos, cierra todos los archivos abiertos, se asegura de que todos los
file descriptors sean cerrados de forma correcta, mediante a los métodos de 'Close' y 'Delete' brindados por el Middleware.

Este mecanismo previene pérdidas de datos, mantiene la consistencia del sistema y asegura el correcto uso de los recursos del sistema operativo.

\subsection{Mecanismos de Control de Fallas}

Se contemplará el desarrollo de mecanismos de control de fallas para aumentar la robustez 
y confiabilidad del sistema. Estos mecanismos estarán orientados a gestionar situaciones 
adversas como caídas de procesos, pérdida de nodos de cómputo o interrupciones inesperadas 
durante la ejecución.  

El objetivo es garantizar, en la medida de lo posible, la continuidad del procesamiento 
y la recuperación parcial de información. En futuras versiones del documento se describirán 
las estrategias de detección, mitigación y recuperación ante fallas.
