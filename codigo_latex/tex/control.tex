\section{Mecanismos de Control}

\subsection{Mecanismos de Control de Sincronización}

En el diseño del sistema distribuido se prevé la incorporación de mecanismos de control 
de sincronización, con el objetivo de evitar problemas asociados a la concurrencia, tales 
como \textit{race conditions} o \textit{deadlocks}.  
Estos mecanismos garantizarán un acceso ordenado a los recursos compartidos y la correcta 
coordinación entre los procesos en ejecución.  

En futuras versiones del documento se detallarán las técnicas específicas adoptadas, 
así como las estrategias de coordinación entre procesos y nodos distribuidos.

\subsection{Mecanismos de Control de Señales}

El sistema incorporará un manejo explícito de señales a fin de asegurar una finalización 
correcta y segura de los procesos. En particular, se controlará la señal \texttt{SIGTERM}, 
permitiendo un \textit{graceful quit} que libere recursos, cierre archivos abiertos y 
mantenga la consistencia del sistema.  

La descripción detallada de las rutinas de manejo de señales y los escenarios contemplados 
se incluirán en próximas iteraciones del documento.

\subsection{Mecanismos de Control de Fallas}

Se contemplará el desarrollo de mecanismos de control de fallas para aumentar la robustez 
y confiabilidad del sistema. Estos mecanismos estarán orientados a gestionar situaciones 
adversas como caídas de procesos, pérdida de nodos de cómputo o interrupciones inesperadas 
durante la ejecución.  

El objetivo es garantizar, en la medida de lo posible, la continuidad del procesamiento 
y la recuperación parcial de información. En futuras versiones del documento se describirán 
las estrategias de detección, mitigación y recuperación ante fallas.
