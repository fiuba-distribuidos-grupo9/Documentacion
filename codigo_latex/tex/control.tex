\section{Mecanismos de Control}

\subsection{Mecanismos de Control de Sincronización}

En el diseño del sistema distribuido cuenta con la incorporación de mecanismos de control 
de sincronización, con el objetivo de evitar problemas asociados a la concurrencia, tales 
como \textit{race conditions} o \textit{deadlocks}. 

Estos mecanismos garantizan un acceso ordenado a los recursos compartidos y la correcta 
coordinación entre los procesos en ejecución.  

\subsection{Mecanismos de Control de Señales y Finalización}

El sistema incorpora un manejo explícito de señales a fin de asegurar una finalización 
correcta y segura de los procesos.

En particular, se controla la señal \texttt{SIGTERM},
permitiendo un \textit{graceful quit} que libera recursos, cierra todos los archivos abiertos, se asegura de que todos los
file descriptors sean cerrados de forma correcta, mediante a los métodos de 'Close' y 'Delete' brindados por el Middleware.

Este mecanismo previene pérdidas de datos, mantiene la consistencia del sistema y asegura el correcto uso de los recursos del sistema operativo.

\subsection{Mecanismos de Control de Fallas}

Para que el sistema sea robusto, se realizó el desarrollo de diversos mecanismos de control de fallas, los cuales
garantizan el determinismo y la confiabilidad del sistema.

En dicha implementación se busca optimizar al máximo estos 2 atributos de calidad, dejando de lado optimizaciones
de rendimiento.

Esta decisión de diseño fue tomada para poder controlar la mayor cantidad de escenarios de falla posible,
de forma segura y cumpliento con todos los criterios definidos por la cátedra.

Un poco mas adelante en el documento, se encuentra el apartado 'Tolerancia a Fallas', donde se detalla en profundidad
todas las implementaciones realizadas.
