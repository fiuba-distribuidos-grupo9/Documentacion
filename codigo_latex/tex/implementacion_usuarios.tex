\section{Implementación del Usuario}

La interacción de los usuarios con el sistema distribuido se realizará mediante un esquema 
multihilo. Para cada usuario que se conecte, se levantarán dos hilos principales:

\begin{itemize}
    \item \textbf{Hilo de envío:} Encargado de transmitir al sistema distribuido 
    todos los archivos necesarios (\texttt{menu\_items.csv}, \texttt{transactions.csv}, 
    \texttt{users.csv}, entre otros). Este hilo asegura que la consulta planteada 
    disponga de la información completa para ser resuelta.
    
    \item \textbf{Hilo de recepción:} Dedicado a permanecer en escucha, esperando 
    la respuesta correspondiente a las consultas solicitadas. Una vez que los nodos 
    de procesamiento finalizan, este hilo recibe los resultados y los entrega al usuario.
\end{itemize}

De esta manera se logra una interacción no bloqueante y escalable.  
La arquitectura está diseñada para soportar múltiples usuarios en simultáneo, 
escalando el número de hilos proporcionalmente a la cantidad de clientes activos.
