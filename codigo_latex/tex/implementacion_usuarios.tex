\section{Implementación del Usuario}

La interacción de los usuarios con el sistema distribuido se realizará mediante un esquema 
secuensial. Para cada usuario que se conecte, se respetera una secuencia específica de ejecución.

\begin{itemize}
    \item \textbf{Etapa de envío de información:} En esta etapa el usuario envía toda la información en base a la que el
    sistema debe operar. Esto incluye todos los datasets a procesar para resolver las 4 consultas.
    
    \item \textbf{Etapa de espera:} El cliente queda a la espera de que el sistema distribuido procese la información y genere las respuestas
    a las consultas.

    \item \textbf{Etapa de recepción de resultados:} Una vez que el sistema ha procesado la información, el usuario recibe las respuestas a las consultas
    solicitadas.
\end{itemize}

De esta manera se logra una interacción segura y escalable.

La arquitectura está diseñada para soportar múltiples usuarios en simultáneo, escalando el número de conexiones establecidas proporcionalmente a la cantidad de clientes activos.
