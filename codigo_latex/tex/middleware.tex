\section{Middleware}

El middleware interno que se utiliza, constituye el pilar de la comunicación del sistema "Coffee Shop Analysis", diseñado para orquestar la interacción entre el server y los múltiples nodos de procesamiento. Su arquitectura se basa en un modelo de Middleware Orientado a Mensajes (MOM), implementado con un bróker central (como por ej. RabbitMQ), tal como se visualiza en el Diagrama de Despliegue. 

La función principal de este middleware es abstraer la complejidad de la comunicación en red, permitiendo que los nodos interactúen de forma desacoplada y asincrónica. En lugar de conexiones directas punto a punto, los nodos se comunican a través de colas y tópicos que se definirán en etapas posteriores.

La implementación de la lógica de conexión se centraliza en el componente `MQConnectionHandler` del paquete `shared`, asegurando un manejo consistente de las comunicaciones. 

Sobre esta capa de mensajería, opera un protocolo de aplicación personalizado que definirá la estructura de los mensajes para el envío de datasets, la transmisión de resultados y el manejo de errores. Este diseño no solo cumple con el requisito de abstraer la comunicación mediante grupos, sino que también le agrega al sistema, escalabilidad y flexibilidad, permitiendo añadir o remover nodos de procesamiento dinámicamente sin alterar la lógica central.

Documentación restante en proceso...