\section{Cronograma real del desarrollo}

\subsection{Diseño}

El desarrollo del diseño se llevó a cabo en un período de dos semanas, 
conformando un equipo de tres integrantes.  

Durante la \textbf{primera semana}, el grupo se enfocó en interpretar en profundidad 
los requisitos planteados en el enunciado, analizar la lógica de las consultas y 
definir el diseño conceptual del sistema. Asimismo, en esta etapa se realizó la 
repartición de tareas entre los integrantes, de manera de optimizar el tiempo de 
ejecución restante.

En la \textbf{segunda semana}, cada integrante se abocó a las actividades asignadas, 
según el siguiente detalle:

\subsection*{Luciano}
\begin{itemize}
    \item Vista Lógica.
    \item Vista de Desarrollo.
    \item Vista Física.
    \item Vista de Procesos.
    \item Diseño inicial/conceptual del Middleware.
    \item Diagrama de Paquetes.
\end{itemize}

\subsection*{Axel}
\begin{itemize}
    \item Escenarios de uso (Casos de Uso).
    \item Diagrama de Secuencia.
    \item Diagrama de Actividades.
    \item Diagrama de Robustez.
    \item DAG (Directed Acyclic Graph).
\end{itemize}

\subsection*{Felipe}
\begin{itemize}
    \item Redacción de las definiciones iniciales.
    \item Creación de los cronogramas.
    \item Descripción detallada de la ejecución de las consultas.
    \item Explicaciones introductorias de los mecanismos de control y protocolos.
\end{itemize}

\newpage

\subsection{Middleware y Coordinación de Procesos}

El desarrollo de la primera versión del sistema distribuido se llevó a cabo en un período de dos semanas y media, 
conformando un equipo de tres integrantes.  

Durante la \textbf{primera semana}, el grupo se enfocó en interpretar en profundidad 
los requisitos planteados en el enunciado, analizar posibles implementaciones para las consultas solicitadas,
definir el lenguaje de programación a utilizar, pactar protocolos e identificar bibliotecas necesarias para el desarrollo.

Asimismo, en esta etapa se realizó la repartición de tareas entre los integrantes, de manera de optimizar el tiempo de 
ejecución restante.

En la \textbf{segunda semana}, cada integrante se abocó a las actividades asignadas, 
según el siguiente detalle:

\subsection*{Luciano}
\begin{itemize}
    \item Implementación del Middleware.
    \item Creación de los tests unitarios para validad funcionamiento del Middleware.
    \item Establecimiento de conexiones entre Cliente y Servidor.
    \item Redacción de protocolos de comunicación.
    \item Integración del Middleware con los controladores.
\end{itemize}

\subsection*{Axel}
\begin{itemize}
    \item Implementación de los controladores:
    \begin{itemize}
        \item Join.
        \item Sorts.
        \item Maps.
        \item Filters.
    \end{itemize}
    \item Integración de los controladores con el Middleware.
    \item Construcción del ecosistema de nodos de cómputo distribuido mediante a Dockerfiles.
    \item Medición del rendimiento del sistema.
\end{itemize}

\subsection*{Felipe}
\begin{itemize}
    \item Implementación de los controladores:
    \begin{itemize}
        \item Cleaners.
        \item Output Builders.
        \item Reduces.
    \end{itemize}
    \item Corrección del diseño inicial según las necesidades del desarrollo y las correcciones indicadas por el docente a cargo.
    \item Redacción de la nueva documentación del sistema.
    \item Implementación de los mecanismos de cierre 'graceful' de los nodos de cómputo.
\end{itemize}
