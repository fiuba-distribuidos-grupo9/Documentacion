\section{Cronograma real del desarrollo}

\subsection{Diseño (2 semanas)}

Durante la \textbf{primera semana}, el grupo se enfocó en interpretar en profundidad 
los requisitos planteados en el enunciado, analizar la lógica de las consultas y 
definir el diseño conceptual del sistema. Asimismo, en esta etapa se realizó la 
repartición de tareas entre los integrantes, de manera de optimizar el tiempo de 
ejecución restante.

En la \textbf{segunda semana}, cada integrante se abocó a las actividades asignadas, 
según el siguiente detalle:

\subsection*{Luciano}
\begin{itemize}
    \item Vista Lógica.
    \item Vista de Desarrollo.
    \item Vista Física.
    \item Vista de Procesos.
    \item Diseño inicial/conceptual del Middleware.
    \item Diagrama de Paquetes.
\end{itemize}

\subsection*{Axel}
\begin{itemize}
    \item Escenarios de uso (Casos de Uso).
    \item Diagrama de Secuencia.
    \item Diagrama de Actividades.
    \item Diagrama de Robustez.
    \item DAG (Directed Acyclic Graph).
\end{itemize}

\subsection*{Felipe}
\begin{itemize}
    \item Redacción de las definiciones iniciales.
    \item Creación de los cronogramas.
    \item Descripción detallada de la ejecución de las consultas.
    \item Explicaciones introductorias de los mecanismos de control y protocolos.
    \item Unión de todas las documentaciones generadas, en el informe entregable.
\end{itemize}

\newpage

\subsection{Middleware y Coordinación de Procesos (2.5 semanas)}

Durante la \textbf{primera semana}, el grupo se enfocó en interpretar en profundidad 
los requisitos planteados en el enunciado, analizar posibles implementaciones para las consultas solicitadas,
hacer las correcciones de diseño necesarias, pactar protocolos e identificar bibliotecas necesarias para el desarrollo.

En la \textbf{segunda semana}, cada integrante se abocó a las actividades asignadas, detalladas a continuación:

\subsection*{Luciano}
\begin{itemize}
    \item Diseño funcional integral de colas y exchanges en el sistema para la topología diseñada.
    \item Investigación y selección de la biblioteca para la comunicación mediante RabbitMQ.
    \item Investigación y selección de la biblioteca para la serialización/deserialización de mensajes.
    \item Implementación del Middleware.
    \item Creación de los tests unitarios para validar funcionamiento del Middleware.
    \item Establecimiento de conexiones entre Cliente, Servidor y Middleware.
    \item Redacción de protocolos de comunicación.
    \item Integración del Middleware con los controladores.
    \item Construcción del ecosistema de nodos de cómputo distribuido mediante a Dockerfiles.
\end{itemize}

\subsection*{Axel}
\begin{itemize}
    \item Diseño funcional integral de colas y exchanges en el sistema para la topología diseñada.
    \item Implementación de los controladores:
    \begin{itemize}
        \item Join.
        \item Sorts.
        \item Maps.
        \item Filters.
    \end{itemize}
    \item Integración de los controladores con el Middleware.
    \item Construcción del ecosistema de nodos de cómputo distribuido mediante a Dockerfiles.
    \item Medición del rendimiento del sistema.
\end{itemize}

\subsection*{Felipe}
\begin{itemize}
    \item Diseño funcional integral de colas y exchanges en el sistema para la topología diseñada.
    \item Implementación de los controladores:
    \begin{itemize}
        \item Cleaners.
        \item Output Builders.
        \item Reduces.
    \end{itemize}
    \item Integración de los controladores con el Middleware.
    \item Redacción de la nueva documentación del sistema.
    \item Relevamiento y control de los mecanismos de cierre 'graceful' de los nodos de cómputo.
\end{itemize}

\subsection{Multi Client (2 semanas)}

Durante la \textbf{primera semana}, repitiendo la operativa realizada en entregas previas, el grupo se enfocó en interpretar en profundidad 
los requisitos planteados en el enunciado, analizar posibles implementaciones para las consultas solicitadas,
hacer las correcciones de diseño necesarias, pactar protocolos e identificar bibliotecas necesarias para el desarrollo.

En la \textbf{segunda semana}, cada integrante se abocó a las actividades asignadas, detalladas a continuación:

\subsection*{Luciano}
\begin{itemize}
    \item Planificación y diseño de la arquitectura multi cliente.
    \item Implementación de la lógica multi cliente adaptada al Servidor.
    \item Implementación de la lógica multi cliente adaptada a los Controladores Statefull.
    \item Redacción del nuevo protocolo de comunicación multi cliente.
    \item Refactorización de los controladores para soportar múltiples sesiones concurrentes.
    \item Adaptación del sistema para configurar la escalabilidad en base al archivo '.env'.
\end{itemize}

\subsection*{Axel}
\begin{itemize}
    \item Planificación y diseño de la arquitectura multi cliente.
    \item Implementación de la lógica multi cliente adaptada al Cliente.
    \item Implementación de la lógica multi cliente adaptada a los Controladores Stateless.
    \item Refactorización de los controladores para soportar múltiples sesiones concurrentes.
    \item Mediciones de rendimiento para múltiples clientes concurrentes.
    \item Redacción de datasets reducidos para la demostración del sistema en la defensa.
\end{itemize}

\subsection*{Felipe}
\begin{itemize}
    \item Planificación y diseño de la arquitectura multi cliente.
    \item Análisis de puntos escalables para optimización del rendimiento.
    \item Redaccion de la nueva documentación del sistema.
    \item Redacción de los nuevos tests de validación funcional de outputs y propagación de EOF.
    \item Redacción de la presentación para la defensa de la entrega.
    \item Relevamiento y control de cumplimiento de los criterios de aceptación.
\end{itemize}

\newpage

\subsection{Paper (1.5 semanas)}

Para esta entrega, en la \textbf{primera semana}, todos los integrantes del grupo realizaron la lectura obligatoria sobre el Paper brindado.
Una vez terminada la misma, se pasó a la redacción del resumen con los conceptos principales que se identificaron para la exposición.

En la \textbf{media semana} restante, se realizó la presentación para la misma, y se practicó varias veces la muestra a realizar.

\subsection*{Luciano}
\begin{itemize}
    \item Lectura completa del Paper.
    \item Distribución de la exposición.
    \item Práctica para la muestra.
\end{itemize}

\subsection*{Axel}
\begin{itemize}
    \item Lectura completa del Paper.
    \item Redacción del resumen.
    \item Práctica para la muestra.
\end{itemize}

\subsection*{Felipe}
\begin{itemize}
    \item Lectura completa del Paper.
    \item Creación de la presentación.
    \item Práctica para la muestra.
\end{itemize}

\newpage

\subsection{Tolerancia a fallas (6.5 semanas)}

Durante la \textbf{primera semana}, siguiendo con la metodología de trabajo adoptada, el grupo se enfocó en interpretar en profundidad 
los requisitos planteados en el enunciado, diseñar posibles implementaciones para diversos escenarios de falla,
hacer las modificaciones de arquitectura necesarias, pactar protocolos e identificar bibliotecas necesarias para el desarrollo.

En la \textbf{segunda semana}, se realizó la validación de las decisiones de diseño tomadas con el corrector, para luego dar paso
a la implementacion de los mecanismos ideados.

Durante las siguientes \textbf{tres semanas} se implementaron todos los sistemas de tolerancia a fallas pactados.

Para finalizar, en la última \textbf{media semana}, se realizó el testo integral el sistema y la documentación final, 
dando así el cierre total al desarrollo del Sistema Distribuido en su completitud.

\subsection*{Luciano}
\begin{itemize}
    \item Planificación y diseño de la arquitectura tolerante a fallas.
\end{itemize}

\subsection*{Axel}
\begin{itemize}
    \item Planificación y diseño de la arquitectura tolerante a fallas.
\end{itemize}

\subsection*{Felipe}
\begin{itemize}
    \item Planificación y diseño de la arquitectura tolerante a fallas.
\end{itemize}
