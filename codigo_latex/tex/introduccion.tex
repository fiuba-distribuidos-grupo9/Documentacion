\section{Introducción}

El presente documento describe el diseño de un sistema distribuido para el análisis de datos de una cadena de cafeterías en Malasia. 
El objetivo principal es procesar grandes volúmenes de información transaccional, de clientes, de sucursales y de productos, 
para obtener métricas clave que apoyen la toma de decisiones de negocio.

De acuerdo con los requisitos existentes, el sistema debe permitir responder las siguientes consultas:

\begin{enumerate}
    \item Listado de transacciones (ID y monto) realizadas entre los años 2024 y 2025, en el rango horario de 06:00 a 23:00, con un monto total mayor o igual a 75.
    \item Identificación de los productos más vendidos (nombre y cantidad) y los que más ganancias han generado (nombre y monto) para cada mes de 2024 y 2025.
    \item Cálculo del \textit{Total Payment Value} (TPV) por cada semestre en 2024 y 2025, discriminado por sucursal, considerando sólo transacciones entre 06:00 y 23:00.
    \item Obtención de la fecha de cumpleaños de los tres clientes con mayor cantidad de compras en 2024 y 2025, para cada sucursal.
\end{enumerate}

Además de los requerimientos funcionales, el sistema deberá cumplir con los siguientes requisitos no funcionales:

\begin{itemize}
    \item Optimización para entornos multicomputadoras, asegurando la escalabilidad ante el crecimiento de datos.
    \item Inclusión de un \textit{middleware} que abstraiga la comunicación entre nodos mediante grupos.
    \item Ejecución única del procesamiento con capacidad de \textit{graceful quit} frente a señales de terminación (SIGTERM).
\end{itemize}

El diseño se realizará bajo un enfoque de arquitectura distribuida, 
buscando flexibilidad, robustez y capacidad de escalar en entornos reales de procesamiento de datos.
