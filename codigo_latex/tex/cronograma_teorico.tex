\section{Cronograma teórico del desarrollo}

\subsection{Diseño}

Previo al inicio del desarrollo efectivo del diseño, el equipo realizó una 
instancia de planificación para organizar la distribución de responsabilidades.  
El objetivo fue asegurar que cada integrante asumiera un conjunto de tareas equilibrado, 
alineado tanto con sus fortalezas como con las necesidades del proyecto.  

La planificación se estructuró en función de las vistas arquitectónicas requeridas, 
los diagramas de modelado, la documentación conceptual y las definiciones de base.

\subsection{Escalabilidad}

El desarrollo de esta entrega cuenta con un tiempo estimado de 2.5 semanas.  
Durante la primera semana se realizará un análisis detallado de los criterios de escalabilidad, 
los requisitos técnicos y el alcance de la entrega, así como la planificación y distribución de tareas 
entre los integrantes del equipo.  

En las semanas posteriores se definirán y ejecutarán las labores específicas de cada integrante, 
las cuales se documentarán en versiones futuras de este informe.  

\subsection{Multi-Client}

El desarrollo de esta entrega cuenta con un tiempo estimado de 2 semanas.  
La primera semana estará destinada al análisis de los criterios, requisitos y alcance de la 
funcionalidad de múltiples clientes, junto con la planificación y asignación de tareas a cada 
integrante del equipo.  

En la semana restante se abordará la ejecución de las labores planificadas, que se detallarán 
en próximas versiones de este documento.  

\subsection{Paper}

El desarrollo de esta entrega cuenta con un tiempo estimado de 1.5 semanas.

Durante la primera semana se llevará a cabo la lectura y análisis del Paper acerca de “Chubby” (Servicio de bloqueo distribuido de Google), abarcando la comprensión de sus conceptos principales, su arquitectura y las conclusiones relevantes para la presentación

Asimismo, se realizará la redacción del documento resumen correspondiente, en el cual se sintetizarán los aspectos técnicos y teóricos más destacados.

El tiempo restante (media semana) será destinado a la preparación de la clase expositiva, incluyendo la elaboración del material de apoyo (Presentación) y la práctica para la exposición de 15 minutos.

\subsection{Tolerancia}

El desarrollo de esta entrega cuenta con un tiempo estimado de 3.5 semanas. 

La primera semana se dedicará al análisis de criterios, requisitos de tolerancia a fallos y alcance 
de la entrega, además de la planificación y distribución de responsabilidades entre los integrantes.  

Una vez sean validadas las implementaciones diseñadas por el corrector, las semanas siguientes estarán
orientadas a la implementación progresiva de los mecanismos planificados.  
