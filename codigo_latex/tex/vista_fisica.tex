\section{Vista Física}

\subsection{Diagrama de Robustez}

El diagrama de robustez forma parte de la vista lógica y sirve como puente entre los casos de uso y el diseño detallado. Este diagrama permite identificar y organizar los principales elementos del sistema, diferenciando entre actores externos, límites de la aplicación y las entidades o controladores que gestionan la lógica de negocio. 

\FloatBarrier
\begin{figure}[H]
  \centering
  \rotatebox{270}{%
    \includegraphics[
      width=.75\textheight,
      keepaspectratio
    ]{img/robustez.png}%
  }
  \caption{Diagrama de Robustez - Completo}
  \label{fig:robustez}
\end{figure}

\newpage

\subsection{Diagrama de Robustez - Primera Consulta}

\FloatBarrier
\begin{figure}[H]
  \centering
  \rotatebox{270}{%
    \includegraphics[
      width=.85\textheight,
      keepaspectratio
    ]{img/robustez1.png}%
  }
  \caption{Diagrama de Robustez - Primera Consulta}
  \label{fig:robustez}
\end{figure}

\newpage

\subsection{Diagrama de Robustez - Segunda Consulta}

\FloatBarrier
\begin{figure}[H]
  \centering
  \rotatebox{270}{%
    \includegraphics[
      width=.85\textheight,
      keepaspectratio
    ]{img/robustez2.png}%
  }
  \caption{Diagrama de Robustez - Segunda Consulta}
  \label{fig:robustez}
\end{figure}

\newpage

\subsection{Diagrama de Robustez - Tercera Consulta}

\FloatBarrier
\begin{figure}[H]
  \centering
  \rotatebox{270}{%
    \includegraphics[
      width=.85\textheight,
      keepaspectratio
    ]{img/robustez3.png}%
  }
  \caption{Diagrama de Robustez - Tercera Consulta}
  \label{fig:robustez}
\end{figure}

\newpage

\subsection{Diagrama de Robustez - Cuarta Consulta}

\FloatBarrier
\begin{figure}[H]
  \centering
  \rotatebox{270}{%
    \includegraphics[
      width=.85\textheight,
      keepaspectratio
    ]{img/robustez4.png}%
  }
  \caption{Diagrama de Robustez - Cuarta Consulta}
  \label{fig:robustez}
\end{figure}

\newpage

\subsection{Escalabilidad de las Operaciones}

Dentro de las operaciones del sistema distribuido, existen tres que admiten escalabilidad 
en la cantidad de nodos de procesamiento:

\begin{itemize}
    \item \textbf{Cleaner}
    \item \textbf{Filter}
    \item \textbf{Join}
\end{itemize}

Como decisión de diseño, se definió que en los casos donde el sistema cuente con una 
disposición de $N$ nodos de alguno de los tres tipos mencionados, comunicándose con 
$M$ nodos de la misma categoría, cada uno de los $M$ nodos receptores tendrá su propia cola.  
Dichas colas recibirán los mensajes enviados por los $N$ nodos anteriores.  
Cuando un nodo de la segunda capa reciba en su cola un mensaje indicador de finalización, 
podrá tener la certeza de que no recibirá más información desde esos nodos, 
y podrá trasladar dicho aviso hacia etapas posteriores en caso de ser necesario.  

Para garantizar este comportamiento, al iniciar la ejecución del sistema todos los nodos 
conocerán la topología de sus vecinos, lo que permitirá coordinar de forma correcta 
la comunicación y el flujo de datos durante el procesamiento.

A continuación, se presenta un esquema ilustrativo de la topología entre $N$ nodos 
\textit{Filter} y $M$ nodos \textit{Join}, junto con sus respectivas $M$ colas:

\FloatBarrier
\begin{figure}[H]
  \centering
  \rotatebox{0}{%
    \includegraphics[
      width=.60\textheight,
      keepaspectratio
    ]{img/nodos_escalabilidad.png}%
  }
  \caption{Esquema representativo de la comunicación de 'N' nodos con 'M' colas de otros 'M' nodos}
  \label{fig:robustez}
\end{figure}

\bigskip

Asimismo, se adoptó la decisión de diseño de no escalar con múltiples computadoras los 
\textit{Join} que se realizan con las tablas de \textit{Menú} y de \textit{Stores}, 
ya que ambas son lo suficientemente pequeñas como para cargarse íntegramente en memoria.  
De esta forma, se mantiene la tabla más pequeña residente en memoria y se la utiliza 
para realizar \textit{joins} eficientes a medida que llegan las transacciones, 
maximizando así la performance.

Tampoco se escalaran con múltiples computadoras las operaciones 'Map', 'Reduce', 'Sort' y 'Output Builder'. Ya que son 'Stateful' y su correcto funcionamiento e implementación se ve muy beneficiado de ejecutarse en una sola instancia de computación, sin perder un rendimiento perceptible.

En contraste, para el caso de los \textit{Join} con la tabla de \textit{Usuarios} 
sí se habilita la escalabilidad a múltiples nodos, dado que el volumen de datos 
puede ser significativamente mayor.  
En este escenario, el listado de usuarios se dividirá entre los $M$ nodos mediante 
una función de \textit{hash}.  
Los nodos que depositen la información de transacciones en las colas aplicarán 
la misma función de \textit{hash} sobre los identificadores de usuario, determinando 
de manera directa a qué cola (y por ende a qué nodo) deben dirigir cada registro 
para asegurar que el \textit{join} se realice correctamente.

\subsection{Diagrama de Despliegue}

Dentro de la vista física realizamos el diagrama de despliegue, el cual ilustra la topología del sistema distribuido en términos de sus nodos computacionales. El diagrama muestra los distintos grupos de nodos especializados (Data Cleaners, Filters, Maps, Joins, etc.) y cómo se interconectan a través de un componente central: el MOM Broker. Esta arquitectura facilita la comunicación asincrónica y el desacoplamiento entre los procesos.

\vspace{1cm}

\FloatBarrier
\begin{figure}[H]
  \centering
  \includegraphics[width=\linewidth, height=.82\textheight, keepaspectratio]{img/Vista Física - Diagrama de Despliegue.drawio.png}
  \caption{Diagrama de Despliegue}
  \label{fig:diagrama_de_despliegue}
\end{figure}

