\section{Tests}
\label{sec:tests}

Con el objetivo de garantizar la correcta operación, consistencia y determinismo del
sistema distribuido desarrollado, se implementó un conjunto integral de pruebas
(\emph{tests}) que abordan distintos niveles de validación funcional y estructural.

En total, el sistema cuenta con \textbf{tres grandes grupos de tests}:

\begin{itemize}
    \item \textbf{Tests de Middleware}
    \item \textbf{Tests de Comparación de Outputs}
    \item \textbf{Tests de Propagación de EOF}
\end{itemize}

A continuación, se describen los objetivos, el alcance y los criterios de validación
de cada grupo de pruebas.

\subsection{Tests de Middleware}

Estos tests se encargan de verificar la \textbf{correcta implementación y funcionamiento
del Middleware} desarrollado sobre la biblioteca de comunicación \texttt{RabbitMQ},
siguiendo la interfaz y las especificaciones brindadas por la cátedra.

El objetivo principal de este conjunto de pruebas es asegurar que la capa de comunicación
entre nodos funcione de manera estable, confiable y conforme al protocolo definido,
tanto para el envío como para la recepción de mensajes.  
Entre los aspectos validados se incluyen:

\begin{itemize}
    \item Correcta creación y vinculación de colas y \emph{exchanges}.
    \item Enrutamiento adecuado de mensajes según el tipo de flujo o dataset.
    \item Manejo de \emph{acknowledgements} y confirmaciones de entrega.
    \item Comportamiento esperado del middleware bajo condiciones de carga y escalado.
    \item Gestión de excepciones.
\end{itemize}

Estos tests fueron desarrollados originalmente para la entrega
\textbf{``Escalabilidad, Middleware y Coordinación de Procesos''}, y sirvieron de base
para validar la estabilidad de la infraestructura de mensajería del sistema.

\subsection{Tests de Comparación de Outputs}

El segundo conjunto de pruebas está orientado a la \textbf{validación funcional de los resultados}
obtenidos por el sistema distribuido, comparando las salidas producidas por las consultas
con los resultados esperados.

Dado que en un entorno distribuido las tareas se dividen entre nodos escalables sin
garantizar un orden específico en la entrega de resultados, fue necesario diseñar
técnicas de comparación \textbf{insensibles al orden} para validar el correcto
comportamiento determinístico del sistema.  
De esta forma, las pruebas se centran en comprobar la equivalencia semántica
de los resultados más allá del orden particular de las filas.

\newpage

\paragraph{Validación por consulta.}

\begin{itemize}
    \item \textbf{Query 1:}  
    Verifica que la cantidad de líneas generadas sea exactamente la misma que la esperada,
    confirmando que el volumen de resultados es correcto.

    \item \textbf{Query 2:}  
    Compara tanto la cantidad de filas como el contenido de cada una,
    sin considerar el orden en que fueron entregadas.  
    De este modo, se valida que los resultados sean completos y correctos, independientemente
    del orden de procesamiento distribuido.

    \item \textbf{Query 3:}  
    Aplica los mismos criterios que en la Query 2, validando que las filas generadas
    contengan la misma información y que no haya pérdidas ni duplicados,
    aún cuando el orden de emisión varíe entre ejecuciones.

    \item \textbf{Query 4:}  
    Verifica que:
    \begin{itemize}
        \item La cantidad total de filas coincida con la esperada.
        \item Exista la misma cantidad de respuestas por cada tienda.
        \item Los clientes identificados como los más compradores por tienda
              coincidan en cantidad de compras.
    \end{itemize}
    No se consideran las diferencias en nombres o fechas de nacimiento, ya que
    en caso de empates el orden de los tres máximos puede variar entre ejecuciones.
    Lo relevante es que los valores reportados correspondan a los tres mayores por tienda,
    garantizando así la corrección lógica de la consulta.
\end{itemize}

Estos tests fueron desarrollados inicialmente para la entrega
\textbf{``Escalabilidad, Middleware y Coordinación de Procesos''} y luego
\textbf{mejorados para la entrega ``MultiClient''}.

\subsection{Tests de Propagación de EOF}

Finalmente, se implementó un conjunto de pruebas específicas para validar el
\textbf{mecanismo de propagación de mensajes \texttt{EOF}} dentro del sistema
\emph{MultiClient}.

El objetivo de estos tests es comprobar que:
\begin{itemize}
    \item Cada controlador reciba correctamente los \texttt{EOF} asociados a los flujos
          que le corresponden, diferenciados por \texttt{session\_id}.
    \item La propagación de \texttt{EOF} se realice de forma completa y ordenada hacia
          los nodos sucesores en la cadena de procesamiento.
    \item No existan pérdidas, duplicados ni propagaciones cruzadas entre sesiones.
\end{itemize}

De esta manera, se valida que el mecanismo de cierre de flujos por sesión funcione
correctamente y que el sistema distribuido finalice el procesamiento de cada cliente
de forma independiente y controlada, sin afectar el estado de los demás flujos activos.

Estos tests fueron desarrollados como parte de la entrega
\textbf{``MultiClient''}, a pedido de la cátedra, para asegurar la correcta
implementación del sistema de detección y propagación de fin de flujo dentro del
entorno multiusuario.
