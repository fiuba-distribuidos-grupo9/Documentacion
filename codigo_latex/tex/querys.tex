\section{Descripción de las Consultas}

El sistema debe ser capaz de responder a las siguientes consultas, de acuerdo con la lógica 
planteada en el enunciado. A continuación, se detalla el paso a paso de cada una de ellas, 
indicando las tablas necesarias, las columnas relevantes y la secuencia de operaciones 
distribuidas.

\subsection{Consulta 1: Transacciones filtradas}
\textbf{Objetivo:} Obtener las transacciones (ID y monto) realizadas durante 2024 y 2025, 
entre las 06:00 y las 23:00 horas, con un monto total mayor o igual a 75.

\begin{itemize}
    \item \textbf{Tablas requeridas:} \texttt{transactions}.
    \item \textbf{Columnas utilizadas:} \texttt{transaction\_id}, \texttt{created\_at}, \texttt{final\_amount}.
    \item \textbf{Pasos:}
    \begin{enumerate}
        \item Filtrar transacciones entre 2024 y 2025, en el rango horario indicado.
        \item Filtrar transacciones con monto mayor o igual a 75.
        \item Generar el reporte y devolverlo al usuario.
    \end{enumerate}
\end{itemize}

\subsection{Consulta 2: Productos más vendidos y más rentables}
\textbf{Objetivo:} Identificar, para cada mes de 2024 y 2025, los productos más vendidos (nombre y cantidad) 
y los productos que más ganancias generaron (nombre y monto).

\begin{itemize}
    \item \textbf{Tablas requeridas:} \texttt{transaction\_items}, \texttt{menu\_items}.
    \item \textbf{Columnas utilizadas:} 
    \begin{itemize}
        \item De \texttt{transaction\_items}: \texttt{item\_id}, \texttt{created\_at}, \texttt{quantity}, \texttt{subtotal}.
        \item De \texttt{menu\_items}: \texttt{item\_id}, \texttt{item\_name}.
    \end{itemize}
    \item \textbf{Pasos:}
    \begin{enumerate}
        \item Filtrar los \texttt{transaction\_items} de 2024 y 2025.
        \item Generar una nueva columna \texttt{year\_month} a partir de la fecha.
        \item Para productos más vendidos:
        \begin{enumerate}
            \item Calcular contador de cantidad (\texttt{item\_counter}).
            \item Reducir por clave \texttt{year\_month, item\_id} sumando las cantidades.
            \item Ordenar por cantidad descendente.
        \end{enumerate}
        \item Para productos más rentables:
        \begin{enumerate}
            \item Calcular monto acumulado (\texttt{total\_amount\_counter}).
            \item Reducir por clave \texttt{year\_month, item\_id} sumando subtotales.
            \item Ordenar por monto descendente.
        \end{enumerate}
        \item Unir con \texttt{menu\_items} para obtener nombres de los productos.
        \item Generar el reporte y devolverlo al usuario.
    \end{enumerate}
\end{itemize}

\subsection{Consulta 3: TPV por semestre y sucursal}
\textbf{Objetivo:} Calcular el \textit{Total Payment Value} (TPV) por cada semestre de 2024 y 2025, 
para cada sucursal, considerando únicamente transacciones realizadas entre 06:00 y 23:00.

\begin{itemize}
    \item \textbf{Tablas requeridas:} \texttt{transactions} y \texttt{stores}.
    \item \textbf{Columnas utilizadas:} \texttt{store\_id}, \texttt{created\_at}, \texttt{final\_amount}.
    \item \textbf{Pasos:}
    \begin{enumerate}
        \item Filtrar transacciones entre 2024 y 2025, dentro del rango horario.
        \item Generar nueva columna \texttt{year\_semester} a partir de la fecha.
        \item Reducir por clave \texttt{year\_semester, store\_id} sumando los montos finales.
        \item Unir con \texttt{stores} para obtener nombres de las sucursales.
        \item Generar el reporte y devolverlo al usuario.
    \end{enumerate}
\end{itemize}

\subsection{Consulta 4: Clientes más frecuentes y cumpleaños}
\textbf{Objetivo:} Obtener la fecha de cumpleaños de los tres clientes con mayor número de compras 
durante 2024 y 2025, discriminado por sucursal.

\begin{itemize}
    \item \textbf{Tablas requeridas:} \texttt{transactions}, \texttt{users} y \texttt{stores}.
    \item \textbf{Columnas utilizadas:} 
    \begin{itemize}
        \item De \texttt{transactions}: \texttt{user\_id}, \texttt{store\_id}, \texttt{created\_at}.
        \item De \texttt{users}: \texttt{user\_id}, \texttt{birthdate}.
    \end{itemize}
    \item \textbf{Pasos:}
    \begin{enumerate}
        \item Filtrar transacciones de 2024 y 2025.
        \item Generar clave combinada \texttt{store\_user} con \texttt{store\_id} y \texttt{user\_id}.
        \item Calcular contador de compras (\texttt{buys\_counter}).
        \item Reducir por clave \texttt{store\_user} sumando los contadores.
        \item Ordenar en forma descendente por número de compras.
        \item Seleccionar los tres usuarios con más compras por sucursal.
        \item Unir con \texttt{users} para obtener las fechas de nacimiento.
        \item Unir con \texttt{stores} para obtener nombres de las sucursales.
        \item Generar el reporte y devolverlo al usuario.
    \end{enumerate}
\end{itemize}
