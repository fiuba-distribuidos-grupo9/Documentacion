\section{Uso de Archivos Intermedios}

En el diseño del sistema se contempla la generación de archivos intermedios durante 
el procesamiento de las consultas. Esta decisión se fundamenta en un aspectos clave:

\textbf{Respaldo de información:} Los archivos intermedios actúan como puntos de control que facilitan futuras implementaciones de tolerancia a fallas. 
En caso de interrupciones, será posible retomar el procesamiento desde un estado parcial previamente almacenado.

El uso de archivos intermedios, si bien introduce un costo adicional de I/O, 
aumenta la robustez del sistema y habilita mejoras posteriores en términos de 
recuperación y confiabilidad.

Para esta entrega del sistema todavía no están implementados de manera funcional, pero si se deja esta consideración para luego documentar su funcionamiento cuando corresponda.