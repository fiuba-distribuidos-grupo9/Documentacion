\section{Uso de Archivos Intermedios}

En el diseño del sistema se contempla la generación de archivos intermedios durante 
el procesamiento de las consultas. Esta decisión se fundamenta en un aspectos clave:

\textbf{Respaldo de información:} Los archivos intermedios actúan como puntos de control que facilitan las implementaciones realizadas para la tolerancia a fallas. 
En caso de interrupciones, hacen posible retomar el procesamiento desde un estado parcial previamente almacenado.

El uso de archivos intermedios, si bien introduce un costo adicional de I/O, 
aumenta la robustez del sistema, garantizando un comportamiento determinístico del sistema, incluso ante diversos escenarios de falla.

Tanto la generación, como el funcionamiento de estos archivos se detalla en profundidad en el apartado 'Tolerancia a Fallas'.
