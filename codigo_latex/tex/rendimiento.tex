\section{Mediciones de rendimiento del sistema implementado}

Para evaluar el rendimiento del sistema distribuido implementado, se realizaron diversas
mediciones bajo diferentes configuraciones y cargas de trabajo a lo largo de todo el desarrollo.

A continuación, se detallan los resultados obtenidos al finalizar el mismo, y su análisis correspondiente.

\subsection{Configuración del entorno de pruebas}

Las pruebas se llevaron a cabo en un entorno controlado, utilizando una red local con
múltiples nodos de cómputo.

Todos los nodos escalables, fueron configurados para levantar 2 nodos de cómputo cada uno.

Respecto a los nodos no escalables, solo se levantó un nodo de cómputo por cada uno (Es decir, para esta entrega no se implementó redundancia en los nodos no escalables, pero no se descarta para futuras versiones del desarrollo).

\subsection{Tiempo de procesamiento total}

Para cumplir con los requisitos brindados por la cátedra, donde se pedia que el sistema tarde menos de una hora en realizar
todo el procesamiento de las 4 consultas, con el dataset completo, se realizó la medición del tiempo total de procesamiento al finalizar
el desarrollo completo de la solución.

El tiempo total de procesamiento medido fue de exactamente '29' minutos con '32' segundos, cumpliendo así con el requisito establecido.

Con esto, se puede concluir que el sistema implementado es capaz de manejar eficientemente el procesamiento de las consultas
dentro del tiempo límite especificado, demostrando su efectividad y capacidad para trabajar con grandes volúmenes de datos
de manera distribuida.
