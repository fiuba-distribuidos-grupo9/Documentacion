\section{Mediciones de rendimiento del sistema implementado}
\label{sec:rendimiento}

Para evaluar el rendimiento del sistema distribuido implementado, se realizaron
diversas mediciones bajo diferentes configuraciones y cargas de trabajo, poniendo
especial énfasis en la capacidad del sistema para manejar múltiples clientes
en ejecución simultánea (\emph{MultiClient}) sobre el dataset completo.

A continuación, se detallan los resultados obtenidos y su análisis correspondiente.

\subsection{Configuración del entorno de pruebas}

Las pruebas se llevaron a cabo en un entorno controlado, utilizando una red local
con varios nodos de cómputo interconectados.  
Durante las mediciones, el sistema fue desplegado de manera que todos los nodos
escalables ajustaran automáticamente su cantidad de instancias según lo considerado
óptimo para mantener un equilibrio entre rendimiento, utilización de recursos y
tiempo de respuesta.  

En otras palabras, no se fijó una cantidad estática de nodos de cómputo por servicio,
sino que el sistema escaló dinámicamente cada componente según la demanda generada
por la cantidad de clientes concurrentes y el volumen de procesamiento requerido
en cada fase de ejecución.

\subsection{Escenarios de prueba}

Se midió el comportamiento del sistema con \textbf{1}, \textbf{2} y \textbf{3 clientes}
en ejecución simultánea, todos procesando el \textbf{dataset completo} y ejecutando
las \textbf{cuatro consultas} definidas por la cátedra.  
Cada cliente opera de manera aislada dentro de su propia sesión (\texttt{session\_id}),
manteniendo así independencia lógica entre los flujos y permitiendo evaluar la
escalabilidad real del sistema frente a cargas concurrentes.

\subsection{Tiempos de procesamiento obtenidos}

Los resultados medidos fueron los siguientes:

\begin{itemize}
    \item \textbf{1 cliente:} El tiempo total de procesamiento fue \textbf{X minutos y X segundos}.
    \item \textbf{2 clientes:} El tiempo total de procesamiento fue \textbf{X minutos y X segundos}.
    \item \textbf{3 clientes:} El tiempo total de procesamiento fue \textbf{X minutos y X segundos}.
\end{itemize}

Se observa que, si bien el tiempo total aumenta con la cantidad de clientes activos,
el incremento no es lineal. El sistema logra mantener una alta eficiencia en la
distribución de tareas gracias a la paralelización de los flujos y al escalado
dinámico de los nodos de cómputo.
