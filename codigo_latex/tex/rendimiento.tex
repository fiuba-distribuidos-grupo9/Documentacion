\section{Mediciones de rendimiento del sistema implementado}
\label{sec:rendimiento}

Para evaluar el rendimiento del sistema distribuido implementado, se realizaron
diversas mediciones bajo diferentes configuraciones y cargas de trabajo, poniendo
especial énfasis en la capacidad del sistema para manejar múltiples clientes
en ejecución simultánea (\emph{MultiClient}) sobre el dataset completo.

A continuación, se detallan los resultados obtenidos y su análisis correspondiente.

\subsection{Configuración del entorno de pruebas}

Las pruebas se llevaron a cabo en un entorno controlado, utilizando una red local
con varios nodos de cómputo interconectados.  
Durante las mediciones, el sistema fue desplegado de manera que todos los nodos ajustaran automáticamente su cantidad de instancias según lo considerado
óptimo para mantener un equilibrio entre rendimiento, utilización de recursos y
tiempo de respuesta.  

En otras palabras, no se fijó una cantidad estática de nodos de cómputo por servicio,
sino que el sistema escaló dinámicamente cada componente según la demanda generada
por la cantidad de clientes concurrentes y el volumen de procesamiento requerido
en cada fase de ejecución.

\subsection{Escenarios de prueba}

Se midió el comportamiento del sistema con \textbf{1}, \textbf{2} y \textbf{3 clientes}
en ejecución simultánea, todos procesando el \textbf{dataset reducido} y ejecutando
las \textbf{cuatro consultas} definidas por la cátedra.

Cada cliente opera de manera aislada dentro de su propia sesión (\texttt{session\_id}),
manteniendo así independencia lógica entre los flujos y permitiendo evaluar la
escalabilidad real del sistema frente a cargas concurrentes.

Además se hizo la medición del tiempo total de procesamiento para un único cliente
ejecutando las cuatro consultas sobre el \textbf{dataset completo}.

\subsection{Dataset completo (100\%) - Único cliente}

El tiempo medido para la ejecución de las cuatro consultas sobre el dataset completo
con un único cliente fue de \textbf{12 minutos y 33 segundos}.

\newpage

\subsection{Dataset reducido (30\%) - Múltiples clientes}

El tiempo medido para la ejecución de las cuatro consultas sobre el dataset reducido para múltiples clientes fue el siguiente:

\begin{itemize}
    \item \textbf{1 cliente:} El tiempo total de procesamiento fue \textbf{3 minutos y 2 segundos}.
    \item \textbf{2 clientes:} El tiempo total de procesamiento fue \textbf{8 minutos y 17 segundos}.
    \item \textbf{3 clientes:} El tiempo total de procesamiento fue \textbf{10 minutos y 57 segundos}.
\end{itemize}

\FloatBarrier
\begin{figure}[H]
  \centering
  \rotatebox{0}{%
    \includegraphics[
      width=.45\textheight,
      keepaspectratio
    ]{img/grafico_clientes.png}%
  }
  \caption{Diagrama de Rendimiento}
  \label{fig:robustez}
\end{figure}

\textbf{Podemos notar lo siguiente:}

\begin{itemize}
    \item \textbf{Crecimiento no lineal:} \\
    El tiempo total no aumenta de forma proporcional al número de clientes. Pasar de 1 a 2 clientes incrementa el tiempo en un \textbf{173\%}, mientras que pasar de 2 a 3 clientes incrementa en \textbf{32\%}. \\
    Esto indica que los recursos del sistema no escalan linealmente con la carga.

    \item \textbf{Efectos de concurrencia y contención:} \\
    A medida que más clientes ejecutan consultas en paralelo, el sistema debe compartir recursos. En nuestro sistema distribuido esto genera \textbf{sobrecarga de sincronización, bloqueos o esperas en colas de ejecución}, lo que degrada el rendimiento.

    \item \textbf{Tendencia de saturación:} \\
    El salto entre 1 y 2 clientes es muy pronunciado, pero entre 2 y 3 el aumento es menor. Esto puede sugerir que el sistema llegó a un \textbf{punto de uso máximo de recursos}, donde agregar más clientes solo incrementa marginalmente el tiempo de respuesta total (todos esperan en cola).
\end{itemize}

\textbf{Conclusión general:} A pesar del incremento en el tiempo con más clientes, el sistema muestra un \textbf{buen rendimiento general} y mantiene tiempos de respuesta aceptables bajo concurrencia. Esto sugiere que la arquitectura utilizada maneja adecuadamente la carga y ofrece un desempeño estable en escenarios con múltiples clientes.

